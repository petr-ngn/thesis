\KOMAoptions{paper=landscape,DIV=last}
\newgeometry{hmargin=2.5cm,bottom=25mm,height=150mm, includehead}
\fancyheadoffset{0pt}
\chapter{Source Codes}


\section{Python Notebook Code (\lstinline{Masters_Thesis.ipynb})}
\begin{lstlisting}[basicstyle=\footnotesize\ttfamily]
import warnings

warnings.filterwarnings("ignore")

import os
import numpy as np
import pandas as pd
import matplotlib
import matplotlib.pyplot as plt
from matplotlib import ticker
import seaborn as sns
import time
import math
import missingno
from itertools import combinations

from scipy.stats import chi2_contingency, ks_2samp, pointbiserialr, somersd
from imblearn.over_sampling import ADASYN

from optbinning import BinningProcess

from sklearn.model_selection import train_test_split, StratifiedKFold
from sklearn.feature_selection import SequentialFeatureSelector
from sklearn.metrics import (
	accuracy_score,
	recall_score,
	precision_score,
	f1_score,
	matthews_corrcoef,
	brier_score_loss,
	confusion_matrix,
	roc_curve,
	roc_auc_score,
	log_loss,
)


from sklearn.linear_model import LogisticRegression
from sklearn.tree import DecisionTreeClassifier
from sklearn.naive_bayes import GaussianNB
from sklearn.ensemble import RandomForestClassifier, GradientBoostingClassifier
from sklearn.neighbors import KNeighborsClassifier
from sklearn.svm import SVC
from sklearn.neural_network import MLPClassifier

from skopt import BayesSearchCV
from skopt.space import Real, Categorical, Integer
import shap

import lime
import dill

data = pd.read_csv("./data/raw_data.csv")
\end{lstlisting}
\subsection{Data Exploration}
\begin{lstlisting}[basicstyle=\footnotesize\ttfamily]
data.info()

no_duplicates = data.duplicated(subset = [col for col in data.columns if col != 'BAD']).sum()
print(f"Duplicates check: {no_duplicates} duplicated rows.")


def categorical_numeric_vars(df: pd.DataFrame, target:str = "BAD") -> tuple[list, list]:
    
    cat_vars = [col for col in df.columns if df[col].dtypes == "O" and col != target]
    num_vars = [col for col in df.columns if col not in cat_vars + [target]]

    print(f"Categorical features: {', '.join(cat_vars)}")
    print(f"Numeric features: {', '.join(num_vars)}")

    return (cat_vars, num_vars)

cat_vars, num_vars = categorical_numeric_vars(data)


def default_distribution_plot(df, target="BAD", export=True):
    
    matplotlib.rcParams["mathtext.fontset"] = "stix"
    matplotlib.rcParams["font.family"] = "STIXGeneral"
    
    # Replacing the Booleans with non-default/default strings for visualization
    df_plot = df[[target]].copy().replace({0: "Non-default", 1: "Default"})

    # Figure's initialization
    fig, ax = plt.subplots(figsize=(8, 6))
    
    # Default distribution
    sns.countplot(data=df_plot, x=target, palette="BuPu",
                  order=list(df_plot[target].unique())[::-1], ax=ax)
    
    # Increase the fontsize for xlabel and ylabel
    ax.set_ylabel(ax.yaxis.label.get_text(), fontsize=17)
    ax.set_xlabel(ax.xaxis.label.get_text(), fontsize=17)

    # Increase the fontsize for xticks and yticks
    plt.xticks(fontsize=17)
    plt.yticks(fontsize=17)

    # Removing upper and right axes spines
    ax.spines["top"].set_visible(False)
    ax.spines["right"].set_visible(False)

    plt.tight_layout()
    
    # Exporting plot
    if export:
        os.makedirs("./plots/", exist_ok=True)
        plt.savefig(f"./plots/Default_Distribution.jpg", dpi=300)

    plt.show()

default_distribution_plot(data)


def numeric_distribution_plot(df: pd.DataFrame, num_vars: list,
                                 plot_type: str, target: str = "BAD", 
                                 export: bool = True):

    def custom_int_formatter(x, pos):
        return f"{x:.0f}"

    #Possible plot types
    plot_types = {"boxplot": sns.boxplot,
                  "violinplot": sns.violinplot,
                  "histogram": sns.histplot}

    #Plot type parameter check
    if plot_type not in plot_types.keys():
		plot_types = ','.join([key for key in plot_types.keys()])
        raise ValueError(f"Invalid plot type. Please, select one of these available plot types: {plot_types}")

        
    matplotlib.rcParams["mathtext.fontset"] = "stix"
    matplotlib.rcParams["font.family"] = "STIXGeneral"
    
    #Plot either boxplots or violinplots
    if plot_type in ["boxplot", "violinplot"]:
        
        #replace 0/1's with (Non)-default texts for visualization's sake.
        df_plot = df.copy()
        df_plot[target] = df_plot[target].replace({0: "Non-default", 1: "Default"})

        #Figure's and axes' initialization
        fig, axs = plt.subplots(nrows = 5, ncols = 2, figsize = (12, 20))

        for ax, var in zip(axs.ravel(), num_vars):
            
            #Boxplot/violinplot
            plot_types[plot_type](data = df_plot, x = target, y = var, ax = ax,
                                  palette = "BuPu", order = ["Non-default", "Default"])

            ax.set_title(f"Distribution of {var}", size = 19, fontweight = "bold")
            ax.tick_params(axis = "both", which = "major", labelsize = 18)
            ax.set(xlabel = None)
            ax.set(ylabel = None)
            ax.spines["top"].set_visible(False)
            ax.spines["right"].set_visible(False)
            ax.yaxis.set_major_formatter(matplotlib.ticker.FuncFormatter(custom_int_formatter))
        
        plt.tight_layout()

        #Exporting the plots
        if export:
            os.makedirs("./plots/", exist_ok = True)
            plt.savefig(f"./plots/Numeric_Features_Distribution_{plot_type.capitalize()}s.jpg", dpi = 300)
        plt.show()


    #Plot histograms
    elif plot_type == "histogram":
        def integer_formatter(x, pos):
            return f"{int(x)}"

        #Figure's and axes' initialization
        fig, axs = plt.subplots(nrows = len(num_vars), ncols = 2, figsize = (12, 25))

        #Column index
        col_ind = 0
        # Axis index
        # (if the value is even, the plot will be located on the left side, otherwise on the right side)
        axis_count = 0

        for ax in axs.ravel():

            #Accessing the feature name.
            var = num_vars[col_ind]
            #Subsetting the data based on the feature.
            var_series = df[var]

            # Calculating the bin size for given feature ensuring that both conditional plots
            # (left and right) will have the same number of bins.
            # Using rule of thumb (Scott, 1979)
            # https://stackoverflow.com/questions/33458566/how-to-choose-bins-in-matplotlib-histogram
       
            R = var_series.max() - var_series.min() #Range of the feature's values
            n = len(var_series) #Number of feature's values
            sigma = var_series.std() #Standard deviation of the feature's values
        
            #Number of bins
            no_bins = int(R*(n**(1/3))/(3.49*sigma))

            # The left side (even axis_count) depicts the features' distribution
            # conditional on the non-default cases.
            if axis_count % 2 == 0:
                #Subsetting the data for non-default cases only.
                df_subset= df.query(f"{target} == 0")
                #Number of missing values within given feature
                no_missings = df_subset[var].isna().sum()
                # Percentage of missing values within subset of given feature
                # (i.e., percentage of missing values of given feature within the non-default cases)
                pct_missings = no_missings/df_subset.shape[0] * 100

                # Histogram with kernel density function
                # binrange to ensure that both left and right plot will have the same data range.
                plot_types[plot_type](data = df_subset, x = var, ax = ax, bins = no_bins,
                                binrange = ((var_series.min(), var_series.max())),
                                kde = True, color = "lightblue")
 
                ax.set_title(f"Distribution of {var} (Non-default cases)", size = 17, fontweight = "bold")
                ax.tick_params(axis = "both", which = "major", labelsize = 15, rotation = 30)
                ax.set(xlabel = None)
                ax.set_ylabel(ax.get_ylabel(), fontsize=15)
                ax.xaxis.set_major_formatter(ticker.FuncFormatter(integer_formatter))
                ax.spines["top"].set_visible(False)
                ax.spines["right"].set_visible(False)

                #Inserting a text box with an information about the missing values.
                ax.text(0.7, 0.9, f"Number of NA's: {no_missings} ({pct_missings:.1f}%)",
                       horizontalalignment = "center", verticalalignment = "center",
                       fontsize = 14,
                       transform = ax.transAxes, bbox = dict(facecolor = "pink", alpha = 0.3))

            # The right side (odd axis_count) depicts the features' distribution
            # conditional on the default cases.
            else:
                #Subsetting the data for default cases only.
                df_subset = df.query(f"{target} == 1")
                #Number of missing values within given feature
                no_missings = df_subset[var].isna().sum()
                # Percentage of missing values within subset of given feature
                # (i.e., percentage of missing values of given feature within the default cases)
                pct_missings = no_missings/df_subset.shape[0] * 100

                # Histogram with kernel density function
                # binrange to ensure that both left and right plot will have the same data range.
                sns.histplot(data = df_subset, x = var, ax = ax, bins = no_bins,
                                binrange = ((var_series.min(), var_series.max())),
                                kde = True, color = "mediumpurple")
                
                ax.set_title(f"Distribution of {var} (Default cases)", size = 17, fontweight = "bold")
                ax.tick_params(axis = "both", which = "major", labelsize = 15, rotation = 30)
                ax.set(xlabel = None)
                ax.set_ylabel(ax.get_ylabel(), fontsize=15)
                ax.xaxis.set_major_formatter(ticker.FuncFormatter(integer_formatter))
                ax.spines["top"].set_visible(False)
                ax.spines["right"].set_visible(False)

                #Inserting a text box with an information about the missing values.
                ax.text(0.7, 0.9, f"Number of NA's: {no_missings} ({pct_missings:.1f}%)",
                        horizontalalignment = "center", verticalalignment = "center",
                        fontsize = 14,
                        transform = ax.transAxes, bbox = dict(facecolor = "pink", alpha = 0.3))
                
                #Proceeding with the next feature
                col_ind += 1

            #Switching to the left/right side of the figure
            axis_count +=1
        
        plt.tight_layout()
        
        #Exporting the plots
        if export:
            os.makedirs("./plots/", exist_ok = True)
            plt.savefig(f"./plots/Numeric_Features_Distribution_{plot_type.capitalize()}s.jpg", dpi = 300)
    
        plt.show()


numeric_distribution_plot(data, num_vars, plot_type = "boxplot")
numeric_distribution_plot(data, num_vars, plot_type = "violinplot")
numeric_distribution_plot(data, num_vars, plot_type = "histogram")


def normality_test(data: pd.DataFrame, num_vars: list) -> pd.DataFrame:

    from scipy.stats import shapiro

    output_df = pd.DataFrame(columns = ["Feature", "Shapiro-Wilk", "Significance", "Result"])
    
    for i, col in enumerate(num_vars):
        temp_df = data.copy()
        temp_df = temp_df[temp_df[col].notna()].copy()

        output_df.loc[i, "Feature"] = col

        shapiro_stat, shapiro_p = shapiro(temp_df[col])
        shapiro_significance_stars = "***" if shapiro_p < 0.01 else "**" if shapiro_p < 0.05 \ 
		                            else "*" if shapiro_p < 0.1 else ""
        shapiro_significance_result = "Normally distributed" if shapiro_p > 0.05 \
                                       else "Not normally distributed"

        output_df.loc[i, "Shapiro-Wilk"] = round(shapiro_stat,4)
        output_df.loc[i, "Significance"] = shapiro_significance_stars
        output_df.loc[i, "Result"] = shapiro_significance_result
        
    return output_df


normality_test(data, num_vars)


def num_na_table(df: pd.DataFrame, num_vars: list, target: str = "BAD") -> pd.DataFrame:
    output_df = pd.DataFrame(index = num_vars,
                      columns = ["# of NA's (Y=0)", "# of NA's (Y=1)",  "% of NA's (Y=0)", "% of NA's (Y=1)"])
    for num in num_vars:
        output_df.loc[num, "# of NA's (Y=0)"] = df.query(f"{target} == 0")[num].isna().sum()
        output_df.loc[num, "# of NA's (Y=1)"] = df.query(f"{target} == 1")[num].isna().sum()

        pct_nd = df.query(f"{target} == 0")[num].isna().sum()/df.query(f"{target} == 0").shape[0] * 100
        pct_d = df.query(f"{target} == 1")[num].isna().sum()/df.query(f"{target} == 1").shape[0] * 100
        output_df.loc[num, "% of NA's (Y=0)"] = pct_nd
        output_df.loc[num, "% of NA's (Y=1)"] = pct_d

    return output_df

num_na_table(data, num_vars)


def categorical_distribution_plot(df: pd.DataFrame, cat_vars: list,
                                  target: str = "BAD", export: bool = True):
    
    #Figure's and axes' initialization
    fig, axs = plt.subplots(nrows = len(cat_vars),ncols = 2, figsize = (11, 10))

    #Column index
    col_ind = 0
    # Axis index
    # (if the value is even, the plot will be located on the left side, otherwise on the right side)
    axis_count = 0
    
    matplotlib.rcParams["mathtext.fontset"] = "stix"
    matplotlib.rcParams["font.family"] = "STIXGeneral"
    
    for ax in axs.ravel():

        #Accessing the feature name
        var = cat_vars[col_ind]
        # Subsetting the data based on the feature with subsequent replacing missing values
        # with N/A's strings (for visualization's sake).
        var_target_df = df[[var, target]].copy().fillna("N/A")
        
        #If the feature has some missing values, put the N/A category at the end of the plot.
        if var_target_df.query(f"{var} == 'N/A'").shape[0] != 0:
            categories = [cat for cat in var_target_df[var].unique() if cat != "N/A"] + ["N/A"]
        else:
            categories = var_target_df[var].unique()

        # The left side (even axis_count) depicts the features' distribution
        # conditional on the non-default cases.
        if axis_count % 2 == 0:

            sns.countplot(data = var_target_df.query(f"{target} == 0"), x = var,
                          ax = ax, order = categories, color = "lightblue")

            ax.set_title(f"Distribution of {var} (Non-default cases)", size = 19, fontweight = "bold")
            ax.tick_params(axis = "both", which = "major", labelsize = 18)
            ax.tick_params(axis = "x", rotation = 30, which = "major", labelsize = 18)
            ax.set(xlabel = None)
            ax.set_ylabel(ax.get_ylabel(), fontsize=15)
            ax.spines["top"].set_visible(False)
            ax.spines["right"].set_visible(False)
                        
        
        # The right side (odd axis_count) depicts the features' distribution
        # conditional on the default cases.
        else:
            
            sns.countplot(data = var_target_df.query(f"{target} == 1"), x = var,
                          ax = ax, order = categories, color = "mediumpurple")

            ax.set_title(f"Distribution of {var} (Default cases)", size = 19, fontweight = "bold")
            ax.tick_params(axis = "both", which = "major", labelsize = 18)
            ax.tick_params(axis = "x", rotation = 30, which = "major", labelsize = 18)
            ax.set_ylabel(ax.get_ylabel(), fontsize=15)
            ax.set(xlabel = None)
            ax.spines["top"].set_visible(False)
            ax.spines["right"].set_visible(False)

            #Proceeding with the next feature
            col_ind += 1

        #Switching to the left/right side of the figure
        axis_count += 1

    plt.tight_layout()

    #Exporting the plots
    if export:
        os.makedirs("./plots/", exist_ok = True)
        plt.savefig(f"./plots/Categorical_Features_Distribution.jpg", dpi = 300)
        
    plt.show()


categorical_distribution_plot(data, cat_vars)


def cat_na_table(df: pd.DataFrame, cat_var: str, target: str = "BAD") -> pd.DataFrame:
    temp_df = df.copy()
    temp_df[cat_var] = temp_df[cat_var].fillna("N/A") 
    output_df = pd.DataFrame(index = list(temp_df[cat_var].unique()),
                             columns = ["# (Y=0)", "# (Y=1)",  "% (Y=0)", "% (Y=1)"])

    for cat in output_df.index:
        output_df.loc[cat, "# (Y=1)"] = temp_df.query(f"{target} == 1 and {cat_var} == '{cat}'").shape[0]
        output_df.loc[cat, "# (Y=0)"] = temp_df.query(f"{target} == 0 and {cat_var} == '{cat}'").shape[0]
        pct_d = round(output_df.loc[cat, "# (Y=1)"]/temp_df.query(f"{target} == 1").shape[0], 4)* 100
        pct_nd = round(output_df.loc[cat, "# (Y=0)"]/temp_df.query(f"{target} == 0").shape[0] , 4)* 100
        output_df.loc[cat, "% (Y=1)"] = pct_d
        output_df.loc[cat, "% (Y=0)"] = pct_nd

    return output_df

cat_na_table(data, "JOB")
cat_na_table(data, "REASON")

def point_biserial_corr_table(data: pd.DataFrame, num_features, target = 'BAD') -> pd.DataFrame:
    
    output_df = pd.DataFrame(columns = ['Feature', 'Point-Biserial Correlation', 'Significance'])
    for i, col in enumerate(num_features):
        
        temp_df = data[data[col].notna()].copy()
        coef, p_value = pointbiserialr(temp_df[target], temp_df[col])
        output_df.loc[i, 'Feature'] = col
        output_df.loc[i, 'Point-Biserial Correlation'] = round(coef, 3)

        significance_stars =  '***' if p_value <= 0.01 else '**' if p_value <= 0.05 else \
		              '*' if p_value <= 0.1 else ' '
        output_df.loc[i, 'Significance'] = significance_stars

    return output_df
    

point_biserial_corr_table(data, num_vars)


def cramer_v_table(data: pd.DataFrame, cat_features, target = 'BAD') -> pd.DataFrame:
    
    output_df = pd.DataFrame(columns = ['Variable #1',  'Variable #2', "Cramer's V", 'Significance'])

    pairs = list(combinations(cat_features + [target], 2))

    pairs =  [sorted(pair) for pair in pairs]

    pairs = sorted(pairs)

    for i, pair in enumerate(pairs):
        temp_df = data.copy()
        temp_df = temp_df[temp_df[pair[0]].notna()]
        temp_df = temp_df[temp_df[pair[1]].notna()]

        cross_tab = pd.crosstab(temp_df[pair[0]], temp_df[pair[1]])
        chi2, p_value, *_ = chi2_contingency(cross_tab, correction = False)

        N = cross_tab.sum().sum()
        k = min(cross_tab.shape)
        cr_v = np.sqrt((chi2/N) / (k-1))

        significance_stars =  '***' if p_value <= 0.01 else '**' if p_value <= 0.05 else \
		               '*' if p_value <= 0.1 else ' '

        output_df.loc[i, 'Variable #1'] = pair[0]
        output_df.loc[i, 'Variable #2'] = pair[1]
        output_df.loc[i, "Cramer's V"] = round(cr_v, 3)

        significance_stars =  '***' if p_value <= 0.01 else '**' if p_value <= 0.05 else \
		              '*' if p_value <= 0.1 else ' '
        output_df.loc[i, 'Significance'] = significance_stars

    return output_df


cramer_v_table(data, cat_vars)


def phi_correlation(var1, var2) -> float:

    cross_tab = pd.crosstab(var1, var2)

    chi2, p_value, *_ = chi2_contingency(cross_tab)
    n = cross_tab.sum().sum() 

    phi = np.sqrt(chi2/n)
    
    return phi


def na_phi_corr_table(data: pd.DataFrame, target = 'BAD') -> pd.DataFrame:

    output_df = pd.DataFrame(columns = ['Feature', 'Phi Correlation', 'Significance'])

    for i, col in enumerate(data.drop(target, axis = 1).columns):
        na_indicator = [1 if pd.isnull(i) else 0 for i in data[col]]
        cross_tab = pd.crosstab(data[target], na_indicator)
        chi2, p_value, *_ = chi2_contingency(cross_tab)
        n = cross_tab.sum().sum() 

        phi = np.sqrt(chi2/n)
        significance_stars =  '***' if p_value <= 0.01 else '**' if p_value <= 0.05 else \
		              '*' if p_value <= 0.1 else ' '

        output_df.loc[i, 'Feature'] = col
        output_df.loc[i, 'Phi Correlation'] = round(phi, 3)
        output_df.loc[i, 'Significance'] = significance_stars

    return output_df

na_phi_corr_table(data)


def spearman_corr_matrix_plot(df: pd.DataFrame, num_var: list, export: bool = True):
    
    #Figure's initialization
    plt.figure(figsize = (16,13))
    matplotlib.rcParams['mathtext.fontset'] = 'stix'
    matplotlib.rcParams['font.family'] = 'STIXGeneral'
    #Corelation matrix heatmap
    heatmap = sns.heatmap(df[num_var].corr(method = "spearman"), vmin = -1, vmax = 1,
                        mask = np.triu(np.ones_like(df[num_var].corr())),
                        annot = True, cmap  = "coolwarm",  fmt = ".3f",
                        xticklabels = num_var, yticklabels = num_var,
                        annot_kws={"size": 19})

    plt.xticks(fontsize = 19)
    plt.yticks(fontsize = 19)

    cbar = heatmap.collections[0].colorbar
    cbar.ax.tick_params(labelsize=19)
    
    plt.tight_layout()

    #Exporting the plots
    if export:
        os.makedirs("./plots/", exist_ok = True)
        plt.savefig(f"./plots/Spearman_Correlation_Matrix_Numeric_Features.jpg", dpi = 300)
    
    plt.show()


spearman_corr_matrix_plot(data, num_vars)


def na_dendrogram(df:pd.DataFrame, export:bool = True, plot_suffix:str = ""):

    matplotlib.rcParams['mathtext.fontset'] = 'stix'
    matplotlib.rcParams['font.family'] = 'STIXGeneral'

    #True/False values indicating whether a column has some NA's.
    na_columns_indicators = df.describe(include = "all").T["count"] < df.shape[0]
    #Filtering the column names having NA's.
    na_columns = df.columns[na_columns_indicators]
    #Subsetting the data based on the NA's columns.
    final_df_dendogram = df[na_columns]

    #Plotting the dendrogram
    missingno.dendrogram(final_df_dendogram, orientation = "top", figsize = (10, 6), fontsize = 11)
    #plt.title("Dendrogram of variables having NA's", size = 13, fontweight = "bold")
    plt.yticks(fontsize=17)
    plt.xticks(rotation=45, fontsize=17)

    plt.tight_layout()

    #Exporting the plot
    if export:
        os.makedirs("./plots/", exist_ok = True)
        plt.savefig(f"./plots/NA_Dendrogram{plot_suffix}.jpg", dpi = 300)
    
    plt.show()

na_dendrogram(data)
na_dendrogram(data[data['BAD'] == 1], plot_suffix = "_defaults")
\end{lstlisting}

\subsection{Data Preprocessing}
\begin{lstlisting}[language=Python, basicstyle=\footnotesize\ttfamily]
	def data_split(
		df: pd.DataFrame,
		test_size: float,
		validation_size: float,
		seed: int,
		num_vars,
		cat_vars,
		oversampling: str = "None",
		target: str = "BAD",
	) -> tuple[pd.DataFrame, pd.Series, pd.DataFrame, pd.Series, pd.DataFrame, pd.Series]:
	
		# Separating the target variable (Y) and the features (X)
		Y = df[target]
		X = df.drop(target, axis=1)
	
		# Stratified split into training set, validation set and test set
		X_temp, X_test, y_temp, y_test = train_test_split(
			X, Y, stratify=Y, test_size=test_size, random_state=seed
		)
		X_train, X_valid, y_train, y_valid = train_test_split(
			X_temp, y_temp, stratify=y_temp, test_size=validation_size, random_state=seed
		)
	
		# ADASYN Oversampling
		if oversampling == "ADASYN":
	
			# ADASYN initialization
			adasyn = ADASYN(random_state=seed, n_jobs=-1)
	
			# Temporary imputing of missing values
			# (separately for categorical and numeric features)
			# ADASYN cannot work with NA's, thus we replace them with arbitrary values.
			X_train_imputed = pd.concat(
				(
					X_train[cat_vars].copy().fillna("NAN"),
					X_train[num_vars].copy().fillna(99999999999999),
				),
				axis=1,
			)[X_train.columns].copy()
	
			# Dummy encoding of categorical features
			# ADASYN cannot work with categorical features, thus we need to encode them.
			for cat in cat_vars:
				X_cat_dummies = pd.get_dummies(X_train_imputed[[cat]])
				X_train_imputed = X_train_imputed.drop(cat, axis=1)
				X_train_imputed = pd.concat((X_train_imputed, X_cat_dummies), axis=1)
	
			# Oversampling on the training set.
			X_train_final, y_train_final = adasyn.fit_resample(
				X_train_imputed, pd.DataFrame(y_train)
			)
	
			# Converting dummies back into the original categorical features.
			for cat in cat_vars:
				X_train_final[cat] = (
					X_train_final.loc[
						:, [col for col in X_train_final.columns if cat in col]
					]
					.idxmax(axis=1)
					.str.replace(f"{cat}_", "")
				)
				X_train_final = X_train_final.drop(
					[col for col in X_train_final.columns if f"{cat}_" in col], axis=1
				)
	
			# Replacing the NA's strings back to NA's with respect to the categorical features.
			X_train_final = X_train_final.replace({"NAN": np.nan})
	
			# Replacing back the missing values (99999999999999) with NA's in such case
			# whether the value is exceeding the maximum value of given numeric feature before imputing.
			for num in num_vars:
				max_value = X_train[num].max()
				X_train_final[num] = X_train_final[num].apply(
					lambda x: np.nan if x > max_value else x
				)
	
			y_train_final = y_train_final[target]
	
			X_train_final = X_train_final.reindex(columns=X_train.columns)
	
			return X_train_final, y_train_final, X_valid, y_valid, X_test, y_test
	
		# No oversampling
		else:
	
			return X_train, y_train, X_valid, y_valid, X_test, y_test


seed = 42
test_size = 0.15
validation_size = test_size/(1 - test_size)

(
    X_train,
    y_train,
    X_valid,
    y_valid,
    X_test,
    y_test,
) = data_split(data, test_size, validation_size, seed, num_vars, cat_vars, oversampling="ADASYN")

(
    X_train_orig,
    y_train_orig,
    X_valid_orig,
    y_valid_orig,
    X_test_orig,
    y_test_orig,
) = data_split(data, test_size, validation_size, seed, num_vars, cat_vars)


def print_data_info(**kwargs):
    
    for name, _set in kwargs.items():

        no_dashes = 50
        no_dashes_name_left = int((no_dashes - len(name) - 1)/2)*"-"
        no_dashes_name_right  = f"{(no_dashes - len(name) - len(no_dashes_name_left) - 2)*'-'}"

        data_info = f"{_set['features'].shape[0]} instances, {_set['features'].shape[1]} features"
        no_dashes_data_info_left = int((no_dashes - len(data_info) - 1)/2)*"-"
        no_dashes_data_info_right = f'{(no_dashes - len(data_info) - len(no_dashes_data_info_left) - 2)*"-"}'
        pct_d = np.sum(_set['target']) / _set['target'].shape[0]*100:
        pct_nd = np.sum(_set['target'] == 0)/ _set['target'].shape[0]*100
        target_info = f"Default: {pct_d:.2f}%  | Non-default: {pct_nd:.2f}%"
        no_dashes_target_info_left = int((no_dashes - len(target_info) - 1)/2)*"-"
        length_dashes_target_info_right = (no_dashes - len(target_info) - len(no_dashes_target_info_left) - 2)
        no_dashes_target_info_right = f'{length_dashes_target_info_right*"-"}'


        print(no_dashes*"-")
        print(f"{no_dashes_name_left} {name} {no_dashes_name_right}")
        print(f"{no_dashes_data_info_left} {data_info} {no_dashes_data_info_right}")
        print(f"{no_dashes_target_info_left} {target_info} {no_dashes_target_info_right}")
        print(no_dashes*"-", "\n")

print_data_info(Training = {'features': X_train, 'target': y_train},
                Validation = {'features': X_valid, 'target': y_valid},
                Test = {'features': X_test, 'target': y_test})


				def woe_binning(
					x_train_set: pd.DataFrame,
					y_train_labels: pd.Series,
					cat_vars: list,
					export: bool = True,
					**kwargs,
				) -> tuple[dict, pd.DataFrame]:
				
					binning_path = "./models/feature_preprocessing"
					final_objects_path = "./models/objects_FINAL"
				
					# Initializing the binning process object.
					bn = BinningProcess(
						variable_names=list(x_train_set.columns),
						categorical_variables=cat_vars
					)
				
					# Fitting the binning on training set.
					bn.fit(x_train_set, y_train_labels)
				
					# DataFrame including binned categories' information.
					bins_woe = pd.DataFrame()
				
					for i in x_train_set.columns:
				
						var = bn.get_binned_variable(i).binning_table.build()
						var = var[(~var["Bin"].isin(["Special", "Totals"]))]
						var["Variable"] = i
				
						bins_woe = pd.concat((bins_woe, var))
				
					bins_woe = bins_woe.loc[:, ~bins_woe.columns.isin(["JS"])]
				
					if export:
						for save_path in [binning_path, final_objects_path]:
							os.makedirs(save_path, exist_ok=True)
							bn.save(f"{save_path}/binning_woe_object.h5")
							bins_woe.to_csv(f"{save_path}/woe_bins.csv",
							   index = False)
				
					# Transforming both training set and test set
					# based on the fitted training binning.
					def woe_bin_transform(bn_fit, x_set):
						x_set_binned = bn_fit.transform(x_set, metric="woe")
						x_set_binned.index = x_set.index
				
						return x_set_binned
				
					X_binned_sets = {"X_train_binned": woe_bin_transform(bn, x_train_set)}
					x_set_copy = X_binned_sets["X_train_binned"].copy()
				
					for col in x_set_copy.columns:
						na_woe = (bins_woe
						         .query('Variable == @col and Bin == "Missing"')["WoE"]
							 .values[0]
							 )
						x_set_copy.loc[x_train_set[col].isna(), col] = na_woe
					X_binned_sets["X_train_binned"] = x_set_copy
				
					for name, x_set in kwargs.items():
						X_binned_sets[f"{name}_binned"] = woe_bin_transform(bn, x_set)
						x_set_copy = X_binned_sets[f"{name}_binned"].copy()
						for col in x_set_copy.columns:
							na_woe = (bins_woe
							         .query('Variable == @col and Bin == "Missing"')
								 ["WoE"]
					           		 .values[0])
							x_set_copy.loc[x_set[col].isna(), col] = na_woe
						X_binned_sets[f"{name}_binned"] = x_set_copy
				
					return (X_binned_sets, bins_woe, bn)

				
X_binned_sets, woe_bins, binning_transformator = woe_binning(X_train, y_train, cat_vars,
                                                             X_valid = X_valid, X_test = X_test)
X_train_binned, X_valid_binned, X_test_binned = (xset for i, xset in X_binned_sets.items())

X_binned_sets_orig, woe_bins_orig, binning_transformator_orig = woe_binning(X_train_orig, y_train_orig,
                                                                            export = False, cat_vars,
                                                                            X_valid = X_valid_orig,
                                                                            X_test = X_test_orig)
X_train_binned_orig, X_valid_binned_orig, X_test_binned_orig = (xset for i, xset in X_binned_sets_orig.items())


def prep_data_export(features: tuple, labels: tuple,
                     ind_sets: tuple = ("Training", "Validation", "Test"),
                     csv_name:str = '') -> pd.DataFrame:

    df_list = []
    
    #Join each pair of features and labels data, assign to it a set indicator,
        # append to the list and then,
        #  transform that list into a data frame (and export it).
    for feat, lab, ind in zip(features, labels, ind_sets):
        
        temp = pd.concat((lab, feat), axis = 1)
        temp["set"] = ind
        df_list.append(temp)
    
    dfs = [df for df in df_list]

    final_df = pd.concat(dfs, axis = 0).sort_index()

    if len(csv_name) != 0:
        os.makedirs("./data/", exist_ok = True)
        final_df.to_csv(f"./data/{csv_name}.csv", index = False)
    
    return final_df


interim = prep_data_export((X_train_binned, X_valid_binned, X_test_binned),
                            (y_train, y_valid, y_test),
                            ("Training", "Validation", "Test"),
                            csv_name = "interim_data")


def categorical_distribution_plot_OS(df_orig: pd.DataFrame, df_os, cat_vars: list,
                                  target: str = "BAD", class_val = 1, export: bool = False):
    
    #Figure's and axes' initialization
    fig, axs = plt.subplots(nrows = len(cat_vars),ncols = 2, figsize = (14, 10))

    #Column index
    col_ind = 0
    #Axis index (if the value is even, the plot will be located on the left side, otherwise on the right side)
    axis_count = 0
    
    matplotlib.rcParams['mathtext.fontset'] = 'stix'
    matplotlib.rcParams['font.family'] = 'STIXGeneral'

    class_title_name = 'Default' if class_val == 1 else 'Non-Default'
    
    for ax in axs.ravel():

        #Accessing the feature name
        var = cat_vars[col_ind]
        # Subsetting the data based on the feature with subsequent replacing missing values#
        # with N/A's strings (for visualization's sake).
        var_target_df_orig = df_orig[[var, target]].copy().fillna("N/A")
        var_target_df_os = df_os[[var, target]].copy().fillna("N/A")
        
        #If the feature has some missing values, put the N/A category at the end of the plot.
        if var_target_df_orig.query(f"{var} == 'N/A'").shape[0] != 0:
            categories_orig = [cat for cat in var_target_df_orig[var].unique() if cat != "N/A"] + ["N/A"]
        else:
            categories_orig = var_target_df_orig[var].unique()

        #If the feature has some missing values, put the N/A category at the end of the plot.
        if var_target_df_os.query(f"{var} == 'N/A'").shape[0] != 0:
            categories_os = [cat for cat in var_target_df_os[var].unique() if cat != "N/A"] + ["N/A"]
        else:
            categories_os = var_target_df_os[var].unique()

        # The left side (even axis_count) depicts the features' distribution
        # conditional on the non-default cases.
        if axis_count % 2 == 0:

            sns.countplot(data = var_target_df_orig.query(f"{target} == {class_val}"), x = var,
                          ax = ax, order = categories_orig, color = "lightblue")

            ax.set_title(f"Distribution of {var} ({class_title_name} cases)", size = 19, fontweight = "bold")
            ax.tick_params(axis = "both", which = "major", labelsize = 18)
            ax.tick_params(axis = "x", rotation = 30, which = "major", labelsize = 18)
            ax.set(xlabel = None)
            ax.set_ylabel(ax.get_ylabel(), fontsize=18)
            ax.spines["top"].set_visible(False)
            ax.spines["right"].set_visible(False)
                        
        
        #The right side (odd axis_count) depicts the features' distribution conditional on the default cases.
        else:
            
            sns.countplot(data = var_target_df_os.query(f"{target} == {class_val}"), x = var,
                          ax = ax, order = categories_os, color = "mediumpurple")

            ax.set_title(f"Distribution of {var} ({class_title_name} cases) - Oversampled",
                         size = 19, fontweight = "bold")
            ax.tick_params(axis = "both", which = "major", labelsize = 18)
            ax.tick_params(axis = "x", rotation = 30, which = "major", labelsize = 18)
            ax.set_ylabel(ax.get_ylabel(), fontsize=18)
            ax.set(xlabel = None)
            ax.spines["top"].set_visible(False)
            ax.spines["right"].set_visible(False)

            #Proceeding with the next feature
            col_ind += 1

        #Switching to the left/right side of the figure
        axis_count += 1

    plt.tight_layout()

    #Exporting the plots
    if export:
        os.makedirs("./plots/", exist_ok = True)
        plt.savefig(f"./plots/Categorical_Features_Distribution_OS_{class_title_name}.jpg", dpi = 300)
        
    plt.show()


categorical_distribution_plot_OS(pd.concat((X_train_orig, y_train_orig), axis = 1),
                                 pd.concat((X_train, y_train), axis = 1),
                                   cat_vars, class_val = 1, export = True)

df_temp = pd.DataFrame(columns = ['Feature', 'Category', '# (Y=1)', '# (Y=1) - Oversampled',
                                  '% (Y=1)','% (Y=1) - Oversampled'])
i = 0
orig_temp_df = pd.concat((X_train_orig, y_train_orig), axis = 1).copy()
temp_df = pd.concat((X_train, y_train), axis = 1).copy()
for cat_var in cat_vars:
    orig_temp_df[cat_var] = orig_temp_df[cat_var].fillna("N/A")
    temp_df[cat_var] = temp_df[cat_var].fillna("N/A")

    for category in orig_temp_df[cat_var].unique():
        df_temp.loc[i, 'Feature'] = cat_var
        df_temp.loc[i, 'Category'] = category

        count_d = orig_temp_df.query(f"{cat_var} == '{category}' and BAD == 1").shape[0]
        count_d_os = temp_df.query(f"{cat_var} == '{category}' and BAD == 1").shape[0]
        df_temp.loc[i, '# (Y=1)'] = count_d
        df_temp.loc[i, '# (Y=1) - Oversampled'] = count_d_os

        pct_d = df_temp.loc[i, '# (Y=1)'] /  orig_temp_df.query("BAD == 1").shape[0]
        pct_d_os = df_temp.loc[i, '# (Y=1) - Oversampled'] /  temp_df.query("BAD == 1").shape[0]
        df_temp.loc[i, '% (Y=1)'] = round(pct_d * 100, 2)
        df_temp.loc[i, '% (Y=1) - Oversampled'] = round(pct_d_os * 100, 2)
        i += 1

df_temp['rel diff'] = df_temp['% (Y=1) - Oversampled'] - df_temp['% (Y=1)']
df_temp.reindex(df_temp['rel diff'].abs().sort_values(ascending=False).index)



def numeric_distribution_plot_OS(df_orig: pd.DataFrame, df_os: pd.DataFrame, num_vars: list,
                                 plot_type: str, target: str = "BAD", class_val = 1,
                                 export: bool = True):

    #Possible plot types
    plot_types = {"boxplot": sns.boxplot,
                  "violinplot": sns.violinplot
                  }

    #Figure's and axes' initialization
    fig, axs = plt.subplots(nrows = len(num_vars),ncols = 2, figsize = (11, 45))

    #Column index
    col_ind = 0
    #Axis index (if the value is even, the plot will be located on the left side, otherwise on the right side)
    axis_count = 0

    matplotlib.rcParams['mathtext.fontset'] = 'stix'
    matplotlib.rcParams['font.family'] = 'STIXGeneral'

    class_title_name = 'Default' if class_val == 1 else 'Non-Default'

    for ax in axs.ravel():

        #Accessing the feature name
        var = num_vars[col_ind]

        # The left side (even axis_count) depicts the features' distribution
        # conditional on the non-default cases.
        if axis_count % 2 == 0:

            plot_types[plot_type](data = df_orig.loc[df_orig[target] == class_val], y = var, ax = ax,
                                  color = "lightblue")

            ax.set_title(f"Distribution of {var} ({class_title_name} cases)", size = 17)
            ax.tick_params(axis = "both", which = "major", labelsize = 15)
            ax.tick_params(axis = "x", rotation = 30, which = "major", labelsize = 15)
            ax.set(xlabel = None)
            ax.set_ylabel(ax.get_ylabel(), fontsize=15)
            ax.spines["top"].set_visible(False)
            ax.spines["right"].set_visible(False)

        #The right side (odd axis_count) depicts the features' distribution conditional on the default cases.
        else:

            plot_types[plot_type](data = df_os.loc[df_os[target] == class_val], y = var, ax = ax,
                                  color = "mediumpurple")

            ax.set_title(f"Distribution of {var} ({class_title_name} cases) - Oversampled", size = 17)
            ax.tick_params(axis = "both", which = "major", labelsize = 15)
            ax.tick_params(axis = "x", rotation = 30, which = "major", labelsize = 15)
            ax.set_ylabel(ax.get_ylabel(), fontsize=15)
            ax.set(xlabel = None)
            ax.spines["top"].set_visible(False)
            ax.spines["right"].set_visible(False)

            #Proceeding with the next feature
            col_ind += 1

        #Switching to the left/right side of the figure
        axis_count += 1

    plt.tight_layout()

    #Exporting the plots
    if export:
        os.makedirs("./plots/", exist_ok = True)
        plt.savefig(f"./plots/Numeric_Features_Distribution_{plot_type.capitalize()}s.jpg", dpi = 300)

    plt.show()


numeric_distribution_plot_OS(pd.concat((X_train_orig, y_train_orig), axis = 1),
                                pd.concat((X_train, y_train), axis = 1), num_vars,
                                plot_type = "violinplot", export = False)

def woe_bins_plot(bins_woe: pd.DataFrame, export: bool = True):
    
    matplotlib.rcParams['mathtext.fontset'] = 'stix'
    matplotlib.rcParams['font.family'] = 'STIXGeneral'
    

    fig, axs = plt.subplots(nrows = 4, ncols = 3, figsize = (13, 22))
    
    for i, ax in zip(bins_woe["Variable"].unique(), axs.ravel()):
    
        temp = bins_woe.loc[bins_woe["Variable"] == i]
        sns.barplot(x = temp.index, y = "WoE", data = temp, ax = ax, palette = "BuPu")
        ax.axhline(y = 0, color = "black", linewidth = 1)

        
        cat = False

        for j in temp["Bin"]:
            if isinstance(j, np.ndarray):
                cat = True
                break
        if cat == False:
            labels = list(temp["Bin"])
 
        elif cat == True:
            temp_bins = [k if type(k) == str else str(list(k)) for k in temp["Bin"]]
            labels = [k..replace("[", "").replace("]", "").replace("'","") for k in temp_bins]
        
        ax.set_title(i, size = 20, fontweight = "bold")
        ax.set_xticklabels(labels, rotation = 75, size = 19)
        ax.set_yticklabels(['%.2f' % y for y in ax.get_yticks()], size = 19)
        ax.set_xlabel(ax.get_xlabel(), fontsize=19)
        ax.set_ylabel(ax.get_ylabel(), fontsize=19)

        ax.spines["top"].set_visible(False)
        ax.spines["right"].set_visible(False)
    

    fig.tight_layout()

    
    if export:
        os.makedirs("./plots/", exist_ok = True)
        plt.savefig(f"./plots/WoE_Distribution.jpg", dpi = 300)

    plt.show()


woe_bins_plot(woe_bins)
\end{lstlisting}
\subsection{Modelling}
\begin{lstlisting}[language=Python, basicstyle=\footnotesize\ttfamily]
models_dict = {
               "LR": LogisticRegression(random_state = seed, n_jobs = -1),
               "DT": DecisionTreeClassifier(random_state = seed),
               "GNB": GaussianNB(),
               "KNN": KNeighborsClassifier(n_jobs = -1), 
               "RF": RandomForestClassifier(random_state = seed, n_jobs = -1),
               "GB": GradientBoostingClassifier(random_state = seed),
               "SVM": SVC(probability = True, random_state = seed),
               "MLP": MLP(random_state = seed)
              }


def bayesian_optimization(model, x_train: pd.DataFrame, y_train:
                          pd.Series, seed: int, objective_function: str = "f1"):
    
    estimator = model

    #Adjustment of max_features hyperparameter within the final model selection.
    max_features = len(x_train.columns)

    #Defining a searching space of possible ranges of hyperparameters, given the model.
    
    if type(model) == type(RandomForestClassifier()):

        search_space = {
                        "n_estimators": Integer(100, 1000),
                        "criterion": Categorical(["gini", "entropy", "log_loss"]),
                        "max_depth": Integer(1, 10),
                        "max_features": Integer(1, max_features),
                        "class_weight": Categorical(["balanced", "balanced_subsample", None]),
                        "bootstrap": Categorical([True, False]),
                        "ccp_alpha": Real(0.000000000001, 0.5, prior="log-uniform"),
                        }



    elif type(model) == type(GradientBoostingClassifier()):

        search_space = {
                        "n_estimators": Integer(100, 1000),
                        "loss": Categorical(["log_loss", "exponential"]),
                        "max_depth": Integer(1, 10),
                        "learning_rate":Real(0.0001, 0.2, prior="log-uniform"),
                        "max_features": Integer(1, max_features),
                        "criterion": Categorical(["friedman_mse", "squared_error"]),
                        }



    elif type(model) == type(LogisticRegression()):

        search_space = {
                        "fit_intercept": Categorical([True, False]),
                        "C": Real(0.000001, 5, prior = "log-uniform"),
                        "penalty": Categorical(["l1", "l2", "none", "elasticnet"]),
                        "solver": Categorical(["lbfgs", "liblinear", "newton-cg", "sag", "saga"]),
                        "class_weight": Categorical(["balanced", None])
                        }

        for solver in ["lbfgs", "liblinear", "newton-cg", "sag", "saga"]:
            if solver == "lbfgs" or solver == "newton-cg":
                search_space["penalty"] = Categorical(["l2", "none"])
            elif solver == "sag":
                search_space["penalty"] = Categorical(["l2", "elasticnet"])
                search_space["l1_ratio"] = Real(0, 1, prior = "uniform")
            elif solver == "liblinear":
                search_space["penalty"] = Categorical(["l1", "l2", "none", "elasticnet"])
                if search_space["penalty"] == "l2":
                    search_space["dual"] = Categorical([True, False])
                if search_space["fit_intercept"]:
                    search_space["intercept_scaling"] = Real(0.1, 100.0, prior="log-uniform")
            else:
                search_space["penalty"] = Categorical(["l1", "l2", "none", "elasticnet"])
                search_space["l1_ratio"] = Real(0, 1, prior = "uniform")
                search_space["solver"] = Categorical(["saga"])




    elif type(model) == type(SVC()):
        search_space = {
                        "C": Real(0.000001, 5, prior = "log-uniform"),
                        "kernel": Categorical(["linear", "poly", "rbf", "sigmoid"]),
                        "degree": Integer(1, 10),
                        "class_weight": Categorical(["balanced", None])
                       }


    elif type(model) == type(MLPClassifier()):
        search_space = {
                        "hidden_layer_sizes": Integer(5, 500),
                        "activation": Categorical(["logistic", "identity", "tanh", "relu"]),
                        "solver": Categorical(["lbfgs", "sgd", "adam"]),
                        "learning_rate": Categorical(["constant", "invscaling", "adaptive"]),
                       }
        


    elif type(model) == type(DecisionTreeClassifier()):
        
        search_space = {
                        "criterion": Categorical(["gini", "entropy"]),
                        "max_depth": Integer(1, 10),
                        "max_features": Integer(1, max_features),
                        }  
        


    elif type(model) == type(GaussianNB()):

        search_space = {
                        "var_smoothing": Real(1e-9, 1e-6, prior = "log-uniform")
                       }



    elif type(model) == type(KNeighborsClassifier()):

        search_space = {
                        'n_neighbors': Integer(5, 20),
                        'weights':  Categorical(['uniform', 'distance']),
                        'metric': Categorical(['minkowski', 'manhattan', 'euclidean']),
                        'p': Integer(1, 5)
                       }

        
    #Initialization of the stratified 10-fold cross validation.
    stratified_cv = StratifiedKFold(n_splits = 10, shuffle = True, random_state = seed)

    # Initialization of the Bayesian Optimization.
    # using the stratified 10-fold cross validation and given model,
    # while maximizing the objective function F1 score.
    bayescv = BayesSearchCV(estimator = estimator,
                            search_spaces = search_space,
                            scoring = objective_function, cv = stratified_cv,
                            n_jobs = -1, n_iter = 50,
                            random_state = seed)

    #Fitting the Bayesian optimization algorithm on the training data.
    bayescv.fit(x_train, y_train)

    #Outputs the model with the best tuned hyperparameters with respect to F1 score.
    return bayescv.best_estimator_


def SFS_feature_selection(x_train:pd.DataFrame, y_train:pd.Series,
                   models_dict:dict, seed:int, objective_function: str = "f1", 
                   sfs_method: str = 'forward', export:bool = True) -> pd.DataFrame:
    
    #Printing output settings and functions
    dashes = 115 * "-"
    count = 1

    def print_FS(name: str, dashes: str,
                 count: int, models_dict: dict):

        print(dashes)
        order = f"{count}/{len(models_dict.keys())}"
        no_dashes_left = int((len(dashes) - len(order) - 1)/2)*"-"
        no_dashes_right  = f"{(len(dashes) - len(order) - len(no_dashes_left) - 2)*'-'}"
        print(f"{no_dashes_left} {order} {no_dashes_right}")
        print(dashes)

        text_ = f"FEATURE SELECTION WITH {name.upper()}"
        
        no_dashes_left_ = int((len(dashes) - len(text_) - 1)/2)*"-"
        no_dashes_right_  = f"{(len(dashes) - len(text_) - len(no_dashes_left_) - 2)*'-'}"
        print(f'{no_dashes_left_} {text_} {no_dashes_right_}')
        print(dashes)
        print(dashes, "\n")

    def print_FS_exec_time(time, number_features: int,
                        selected_feats: list, dashes: str):

        print(f'    Execution time: {round(time, 4)} minutes', "\n")
        print(f'    {number_features} features selected: {", ".join(selected_feats)}', "\n")
        print(dashes)
        print(dashes, "\n")
        print()


    #Definitions of paths for exported files    
    opt_models_path = "./models/feature_selection/opt_models"
    feat_select_path = "./models/feature_selection/feat_select_objects"
    df_feat_select_path = "./models/feature_selection"
    
    # Empty list for storing all the models with their tuned hyperparameters,
    # number of features selected, and names of selected features.
    models_feats_list = []

    # For each model, tune its hyperparameters using Bayesian Optimization.
    # Then use the tuned model for feature selection using Forward Sequential Feature Selector,
    # while maximizing an objective function F1 Score.
    for model_name, model in models_dict.items():

        print_FS(model_name, dashes, count, models_dict)

        start = time.time()
        print("    1/4 ... Starting Bayesian Optimization on the whole set of features")

        #Tuning the hyperparameters of the model using Bayesian Optimization.
        tuned_model = bayesian_optimization(model, x_train, y_train, seed, objective_function)

        print("    2/4 ... Bayesian Optimization finished")

        #Initialization of the stratified 10-fold cross validation.
        stratified_cv = StratifiedKFold(n_splits = 10, shuffle = True, random_state = seed)

        print(f"    3/4 ... Starting {sfs_method.capitalize()} Sequential Feature Selection")

        # Initialization of the Forward Sequential Feature Selector 10-fold Cross Validation
        # in order to select the optimal number of features
        # based on the model with tuned hyperparameters and maximizing F1 score function.
        # Instead of selected a fixed number of features, we use an auto selection which stops adding features 
        # when the objective function (F1) is not incremented by at least "tol"
        # between two consecutive feature additions.
                
        sfs = SequentialFeatureSelector(tuned_model, n_features_to_select = "auto",
                                        direction = sfs_method, scoring = objective_function,
                                        cv = stratified_cv, n_jobs = -1, tol = 0.0000000000000000001)
        sfs.fit(x_train, y_train)

        print(f"    4/4 ... {sfs_method.capitalize()} Sequential Feature Selection with finished", "\n")
        end = time.time()

        #Extracting the final selected features and number of the selected features.
        selected_feats = x_train.columns[sfs.get_support()].tolist()
        number_features = len(selected_feats)

        print_FS_exec_time((end - start)/60, number_features, selected_feats, dashes)

        #Exporting both fitted tuned model and feature selection object.
        if export:
            for save_path, export_obj, save_name in zip([opt_models_path, feat_select_path],
                                                        [tuned_model, sfs],
                                                        ["FS_tuned", "SFS_with"]):
        
                os.makedirs(save_path, exist_ok = True)
                with open(f'{save_path}/{save_name}_{model_name}.h5', "wb") as save_obj:
                    dill.dump(export_obj, save_obj)

        #Appeding all the model's information from both Bayesian Optimization an Sequential Feature Selection.
        models_feats_list.append([model_name, #name of the tuned base model
                                  number_features, #number of selected features
                                  selected_feats, #list of selected features' names
                                  (end - start)/60 #execution time
                                  ])

        count += 1
        
    # Storing all the models with their tuned hyperparameters,
    # number of features selected, and names of selected features.
    feat_sel_cols = ["model_name",
                     "n_features", "final_features", "execution_time"]

    feat_sel = pd.DataFrame(models_feats_list, columns = feat_sel_cols)
    
    # Exporting the data frame storing all the models' information
    # from both Bayesian Optimization an Sequential Feature Selection.
    if export:
        with open(f"{df_feat_select_path}/feat_selection_df_ORIGINAL.pkl", "wb") as feat_select_df_save:
            dill.dump(feat_sel, feat_select_df_save)

    return feat_sel


feat_selection_df = SFS_feature_selection(X_train_binned, y_train, models_dict, seed)


def selected_features_recurrence(df: pd.DataFrame, export: bool = True):

    matplotlib.rcParams['mathtext.fontset'] = 'stix'
    matplotlib.rcParams['font.family'] = 'STIXGeneral'
    
    df_plot = (
                pd.get_dummies(df["final_features"]
                               .apply(pd.Series)
                               .stack()
                              )
                .sum(level = 0)
                .sum()
                .reset_index()
                .rename(columns = {"index": "Model", 0: "Count"})
                .sort_values("Count", ascending = False)
              )
    
    fig, ax = plt.subplots(nrows = 1, ncols = 1, figsize = (11, 7))

    
    
    df_plot.plot.bar(x = "Model", y = "Count", color = "mediumpurple",
                     legend = False, ax = ax, zorder = 2)
    
    ax.grid(which = "major", axis = "y", linestyle = "--", zorder = 0)
    
    # set title and axis labels
    #plt.title("Recurrence of the selected features", size = 13, fontweight = "bold")
    #plt.xlabel("Features", size = 14)
    plt.xlabel(None)
    plt.ylabel("Count", size = 20)
    plt.yticks(range(1, df_plot['Count'].max() + 1), size = 20)
    plt.xticks(size = 20, rotation = 45)
    
    plt.legend().set_visible(False)
    
        #Removing upper and right axes spines
    axes = plt.gca()
    axes.spines["top"].set_visible(False)
    axes.spines["right"].set_visible(False)

    plt.tight_layout()

    if export:
        os.makedirs("./plots/", exist_ok = True)
        plt.savefig(f"./plots/Recurrence_Selected_Features.jpg", dpi = 300)

    plt.show()


selected_features_recurrence(feat_selection_df)

def selected_features_dist(df: pd.DataFrame, export: bool = True):
    
    matplotlib.rcParams['mathtext.fontset'] = 'stix'
    matplotlib.rcParams['font.family'] = 'STIXGeneral'
    
    df_plot = (
        pd.concat(
            (
                df[["model_name"]],
                pd.get_dummies(
                    df["final_features"]
                    .apply(pd.Series)
                    .stack()
                )
                .sum(level=0)
            ),
            axis=1
        )
        .copy()
    )

    df_plot["totals"] = df_plot.drop("model_name", axis=1).sum(axis=1)
    df_plot = (
        df_plot
        .sort_values(by="totals", ascending=False)
        .drop("totals", axis=1)
    )

    fig, ax = plt.subplots(nrows=1, ncols=1, figsize=(11, 7))
    
    colors = plt.cm.tab20.colors
    
    df_plot.plot(
        x="model_name",
        y=[col for col in df_plot.columns if col != "model_name"],
        kind="bar",
        stacked=True,
        figsize=(10, 6),
        ax=ax,
        color=colors,
        zorder=3
    )

    ax.grid(which="major", axis="y", linestyle="--", zorder=0)
    #plt.title("Distribution of Selected Features", size=15, fontweight="bold")
    plt.xlabel(None)
    plt.ylabel("Number of selected features", size=19)
    plt.xticks(size=19, rotation=0)
    plt.yticks(range(1,
                     df_plot[[col for col in df_plot.columns if col != "model_name"]].sum(axis=1).max() + 1),
               size=19)
    plt.legend(bbox_to_anchor=(1, 1), fontsize=14)

    # Removing upper and right axes spines
    axes = plt.gca()
    axes.spines["top"].set_visible(False)
    axes.spines["right"].set_visible(False)

    plt.tight_layout()

    if export:
        os.makedirs("./plots/", exist_ok=True)
        plt.savefig(f"./plots/Selected_Features_Distribution.jpg", dpi=300)

    plt.show()


selected_features_dist(feat_selection_df)


def remove_duplicated_selected_feats(df:pd.DataFrame, export:bool = True) -> pd.DataFrame:

   df_feat_select_path = "./models/feature_selection"
   
   #Filtering indices of duplicated selected features
   feat_duplicated_ind = (
                            df["final_features"]
                           .apply( lambda x: ", ".join(x))
                           .duplicated(keep = False)
                           .values
                          )
   
   if sum(feat_duplicated_ind) != 0:
      #Keeping the model which was the fastest to execute and drop the others
      duplicated_models = (
                           df[feat_duplicated_ind]
                           .sort_values("execution_time")
                           .reset_index(drop = True)
                        )
    
      drop_models = (
                   duplicated_models
                   .loc[1:,'model_name']
                   .values
                  )
      
      keep_model = duplicated_models.head(1)["model_name"].values[0]

      models_name_join = ', '.join(duplicated_models["model_name"].tolist())
      
      final_feat_selection_df = (
                                df[~df["model_name"].isin(drop_models)]
                                .reset_index(drop = True)
                                )
      filter_row = (final_feat_selection_df["model_name"] == keep_model)
      final_feat_selection_df.loc[filter_row, "model_name"] = models_name_join
      
   else:
      final_feat_selection_df = df.copy()

   if export:
      with open(f"{df_feat_select_path}/feat_selection_df_FINAL.pkl", "wb") as feat_select_df_save:
         dill.dump(final_feat_selection_df, feat_select_df_save)

   return final_feat_selection_df


feat_selection_df_FINAL = remove_duplicated_selected_feats(feat_selection_df)


metrics_weights = {
                    "descending_rank": {
                                        "F1": 1.5,
                                        "Recall": 1.2,
                                        "Precision": 1,
                                        "Accuracy": 1,
                                        "AUC": 1,
                                        "Somers D": 1,
                                        "KS": 1,
                                        "MCC": 1
                                       },

                    "ascending_rank": {
                                        "Brier Score Loss": 1,
                                        "Log Loss": 1
                                       }
                    }


def calc_opt_threshold(model, x:pd.DataFrame, y:pd.Series) -> float:

    y_scores = model.predict_proba(x)[:, 1]

    fpr, tpr, thresholds = roc_curve(y, y_scores)

    youden_index = tpr + (1 - fpr) - 1
    threshold = thresholds[np.argmax(youden_index)]

    return threshold



def model_selection(x_train:pd.DataFrame, y_train:pd.Series,
                    x_val:pd.DataFrame, y_val:pd.Series,
                    models_dict:dict, feat_sel:pd.DataFrame,
                    seed:int, metrics_weights:dict,
                    target:str = "BAD", objective_function: str = "f1",
                    export:bool = True) -> pd.DataFrame:
    

    #Printing output settings functions
    dashes = 115 * "-"
    count = 1

    def print_model_selection(name: str, fs_name: str, feat_sel: pd.DataFrame,
                              models_dict: dict, dashes: str, count: int):
        
        print(dashes)

        order = f"{count}/{len(models_dict.keys())*feat_sel.shape[0]}"
        no_dashes_left = int((len(dashes) - len(order) - 1)/2)*"-"
        no_dashes_right  = f"{(len(dashes) - len(order) - len(no_dashes_left) - 2)*'-'}"

        print(f"{no_dashes_left} {order} {no_dashes_right}")

        print(dashes)

        text_ = f"BAYESIAN OPTIMIZATION OF {name.upper()}"
        text__ = f"WITH FEATURES SELECTED BY {fs_name.upper()}"
            
        no_dashes_left_ = int((len(dashes) - len(text_) - 1)/2)*"-"
        no_dashes_right_  = f"{(len(dashes) - len(text_) - len(no_dashes_left_) - 2)*'-'}"
        no_dashes_left__ = int((len(dashes) - len(text__) - 1)/2)*"-"
        no_dashes_right__  = f"{(len(dashes) - len(text__) - len(no_dashes_left__) - 2)*'-'}"     

        print(f"{no_dashes_left_} {text_} {no_dashes_right_}")
        print(f"{no_dashes_left__} {text__} {no_dashes_right__}")
        print(dashes)
        print(dashes, "\n")
    

    def exec_time_model_selection(name: str, opt_mod, time, evs_list: list,
                                  threshold: float, dashes: str):
        
        print(f"    Execution time: {round(time, 4)} minutes", "\n")
        print(f"    {objective_function.capitalize()} Score on Validation set: {evs_list[0]}", "\n")
        print(f"    Optimal classification threshold: {round(threshold, 4)}", "\n")
        print(f"    Tuned hyperparameters of {name}:", "\n")
        for hyp, val in opt_mod.get_params().items():
            print(f"        {hyp}: {val}")
        print("")
        print(dashes, "\n")
    
    #Path to save the model selection data frame.
    model_selection_path = "./models/model_selection"

    #Metrics space.
    metrics = { 
                "F1": f1_score,
                "Precision": precision_score, 
                "Recall": recall_score, 
                "Accuracy": accuracy_score,
                "AUC": roc_auc_score,
                "Somers D": somersd,
                "KS": ks_2samp, 
                "MCC": matthews_corrcoef,
                "Brier Score Loss": brier_score_loss,
                "Log Loss": log_loss
                }
    
    #Metrics categorization
    probs_evs = ["AUC", "Brier Score Loss"]
    class_evs = ["Precision", "Recall", "F1", "Accuracy", "MCC"]
    
    #List for storing the results of the model selection.
    tuned_list = []


    #For each model, optimize it on the each subset of optimal features on training set
        # and then evaluate it on validation set with filtered features (using set of several metrics).
    for model_name, model in models_dict.items():
        
        for _, row in feat_sel.iterrows():
            
            #Model name and feature selection method.
            fs_name = row["model_name"]

            #Final features
            final_features = row["final_features"]

            #Filtered training and validation sets based on the final features.
            X_train_filtered = x_train[final_features]
            X_val_filtered = x_val[final_features]

            print_model_selection(model_name, fs_name, feat_sel, models_dict, dashes, count)
            n_f = len(final_features)
            print(f"    1/2 ... Starting Bayesian Optimization on the subset of features ({n_f} features):")
            print(f'               {", ".join(final_features)}')

            start = time.time()

            #Bayesian Optimization
            tuned_model = bayesian_optimization(model, X_train_filtered, y_train, seed, objective_function)

            end = time.time()

            #Optimal threshold calculation
            threshold = calc_opt_threshold(tuned_model, X_train_filtered, y_train)
            
            #Predicted probabilities and classes on validation set (using optimal threshold)
            y_val_scores = tuned_model.predict_proba(X_val_filtered)[:, 1]
            y_val_preds = pd.Series(y_val_scores).apply(lambda x: 1 if x > threshold else 0)

            #List for storing calculated evaluation metrics.
            evs_list = []

            #Metrics calculation on validation set.
            for metric_name, metric in metrics.items():
                if metric_name in probs_evs:
                    evs_list.append(metric(y_val, y_val_scores))
                elif metric_name in class_evs:
                    evs_list.append(metric(y_val, y_val_preds))
                elif metric_name == 'Log Loss':
                    evs_list.append(metric(y_val, tuned_model.predict_proba(X_val_filtered)))
                elif metric_name == "Somers D":
                    evs_list.append(metric(y_val, y_val_scores).statistic)
                elif metric_name == "KS":
                    X_Y_concat = pd.concat((y_val, X_val_filtered), axis = 1)
                    X_Y_concat["prob"] =  y_val_scores
                    evs_list.append(metric(X_Y_concat.loc[X_Y_concat[target] == 1, "prob"],
                                           X_Y_concat.loc[X_Y_concat[target] == 0, "prob"]).statistic)

            #Storing the results of the model selection.
            tuned_list.append([model_name,
                               fs_name,
                               tuned_model,
                               len(final_features),
                               final_features,
                               (end - start)/60,
                               threshold] + 
                               evs_list
                               )

           #Exporting the final tuned model in .h5 format 
            if export:
                os.makedirs(model_selection_path, exist_ok = True)
                os.makedirs(f"{model_selection_path}/models", exist_ok = True)

                with open(f"{model_selection_path}/models/{model_name}_with_{fs_name}.h5", "wb") as mod_save:
                    dill.dump(tuned_model, mod_save)

            
            print("    2/2... Bayesian Optimization finished", "\n")

            exec_time_model_selection(model_name, tuned_model, (end - start)/60, evs_list, threshold, dashes)

            count += 1

    #Model selection data frame.
    model_sel_cols = ["tuned_model_name", "fs_model_name",
                      "sfs_object", "n_features",
                      "final_features", "execution_time",
                      "threshold"] + list(metrics.keys())
    
    model_selection_df = pd.DataFrame(tuned_list, columns = model_sel_cols)
            
    #Ranking the models based on the evaluation metrics and the provided weights.
    for dict_name, metric_weight_dict in metrics_weights.items():

        #Ranking each metric.
        for metric in metric_weight_dict.keys():
                
                # Ranking the models whether the metric is ascending or descending
                # (score metrics are descending, loss metrics are ascending).
                order = True if dict_name == "ascending_rank" else False

                #Ranking
                model_selection_df[f"{metric}_rank"] = (
                                                        model_selection_df[[metric]]
                                                        .rank(ascending = order)
                                                        )
    
    #Accessing individual ranking metrics columns.
    ranked_cols = [rank for rank in model_selection_df.columns if "_rank" in rank]

    unnested_metric_weights = {k: v for in_dict in metrics_weights.values() for k, v in in_dict.items()}


    for i, row in model_selection_df[ranked_cols].iterrows():

        numerator = sum([row[col] * unnested_metric_weights[col.split("_")[0]] for col in ranked_cols])
        deniminator = sum(unnested_metric_weights.values())

        model_selection_df.loc[i, "avg_rank"] = numerator / deniminator
        

    model_selection_df['final_rank'] = model_selection_df['avg_rank'].rank(ascending = True)

    #Ordering the model selection data frame based on the final ranking.
    model_selection_df = (
                           model_selection_df
                          .drop(ranked_cols, axis = 1)
                          .sort_values("final_rank")
                          .reset_index(drop = True)
                          )
    
    #Exporting the model selection data frame.
    if export:
        with open(f"{model_selection_path}/model_selection_df.pkl", "wb") as model_select_df_save:
            dill.dump(model_selection_df, model_select_df_save)

    return model_selection_df



model_selection_df = model_selection(X_train_binned, y_train, X_valid_binned, y_valid,
                                     models_dict, feat_selection_df_FINAL, seed, metrics_weights)


(
    model_selection_df
    .loc[(model_selection_df
          .loc[:, ~model_selection_df
               .columns
               .isin(["final_features"])]
           .groupby("tuned_model_name")["final_rank"]
           .idxmin()), 
         ~model_selection_df
         .columns
         .isin(["final_features"])]
    .sort_values("final_rank")
)


def plot_models_metrics(df: pd.DataFrame, metric: str, group_by: str = "tuned_model_name",
                        plot_name: str = "", export:bool = True):

    matplotlib.rcParams["mathtext.fontset"] = "stix"
    matplotlib.rcParams["font.family"] = "STIXGeneral"

    plt.figure(figsize = (10, 6))  # Increase the figsize values to make the plot bigger
    sns.boxplot(data = df, x = metric, y = group_by, palette = "BuPu_r")

    axes = plt.gca()
    axes.spines["top"].set_visible(False)
    axes.spines["right"].set_visible(False)
    axes.set(ylabel = None)

    # Increase the fontsize for xticks and yticks
    plt.xticks(fontsize = 17)
    plt.yticks(fontsize = 17)

    # Increase the fontsize for x and y axis labels
    if metric == "execution_time":
        plt.xlabel(f"Execution time (in minutes)", fontsize = 17)
    else:
        plt.xlabel(metric, fontsize = 17)

    plt.tight_layout()
    
    if export:
        os.makedirs("./plots/", exist_ok=True)
        plot_name_save = metric.upper() if len(plot_name) == 0 else plot_name.upper()
        plt.savefig(f"./plots/{plot_name_save}_Distribution.jpg", dpi=300)

    plt.show()


plot_models_metrics(model_selection_df, "F1")
plot_models_metrics(model_selection_df[model_selection_df["F1"]>model_selection_df["F1"].min()],
                    "F1", plot_name = "F1_wo_outliers")

plot_models_metrics(model_selection_df, "threshold")
plot_models_metrics(model_selection_df[model_selection_df["threshold"] < model_selection_df["threshold"].max()],
                    "threshold", plot_name = "threshold_wo_outliers")

plot_models_metrics(model_selection_df, "execution_time")

for col in ["Precision", "Recall", "Accuracy", "AUC", "Somers D", "KS", "MCC", "Brier Score Loss", "Log Loss"]:
    plot_models_metrics(model_selection_df, col, export = "_".join([col]))


def scatter_metrics_plot(df: pd.DataFrame, metric_1: str, metric_2: str,
                         plot_name: str = "", export: bool = True):
    
    matplotlib.rcParams["mathtext.fontset"] = "stix"
    matplotlib.rcParams["font.family"] = "STIXGeneral"
    
    plt.figure(figsize = (10, 7))
    plot_df = df.copy()
    sns.scatterplot(data = plot_df, x = metric_1, y = metric_2,
                    hue = "tuned_model_name", palette = "Set2", s = 100, alpha = 0.7)
    axes = plt.gca()
    axes.spines["top"].set_visible(False)
    axes.spines["right"].set_visible(False)

    # Increase the fontsize for xticks and yticks
    plt.xticks(fontsize=18)
    plt.yticks(fontsize=18)

    if metric_1 == "execution_time":
        metric_xlabel = "Execution time (in minutes)"
    else:
        metric_xlabel = metric_1

    if metric_2 == "execution_time":
        metric_ylabel = "Execution time (in minutes)"
    else:
        metric_ylabel = metric_2

    # Increase the fontsize for x and y axis labels
    plt.xlabel(metric_xlabel, fontsize = 18)
    plt.ylabel(metric_ylabel, fontsize = 18)

    plt.legend(bbox_to_anchor = (1.2, 1.1), fontsize = 18)  # Increase the fontsize for the legend
    plt.tight_layout()

    if export:
        os.makedirs("./plots/", exist_ok = True)
        plt.savefig(f"./plots/Scatterplot_{metric_1}_{metric_2}{plot_name}.jpg", dpi = 300)
    
    plt.show()


scatter_metrics_plot(model_selection_df, "execution_time", "F1")
scatter_metrics_plot(model_selection_df[model_selection_df["F1"] > model_selection_df["F1"].min()],
                     "execution_time", "F1", plot_name = "_wo_outliers")


def print_final_model(df: pd.DataFrame, model_order: int = 0):

    final_model_path = "./models/model_selection/models"

    with open(os.path.join(final_model_path, f"{final_model_name}_with_{fs_model_name}.h5"), "rb") as mod_load:
        final_model = dill.load(mod_load)

    final_model_name = df.loc[model_order, "tuned_model_name"]
    fs_model_name = df.loc[model_order, "fs_model_name"]
    final_features = df.loc[model_order, "final_features"]
    final_threshold, final_hyperparameters = df.loc[model_order, "threshold"], final_model.get_params()

    #Final model

    print(f"Final model: {final_model_name.upper()}")
    print(f"Final model trained on features selected by: {fs_model_name.upper()}")
    print(f"Final subset of features: {', '.join(final_features)}")
    print(f"The final threshold: {final_threshold}")
    print("Final hyperparameters:")

    for hp_name, hp_value in final_hyperparameters.items():
        print(f"{(len('Final hyperparameters:') + 2) * ' '}{hp_name}: {hp_value}")


print_final_model(model_selection_df)


def data_filter_join(hyp_tuning_df: pd.DataFrame, x_train: pd.DataFrame,
                     x_valid: pd.DataFrame, x_test: pd.DataFrame,
                     y_train: pd.Series, y_valid: pd.Series,
                     model_order:int = 0) -> tuple[pd.Series, pd.DataFrame, pd.DataFrame]:
    
    #Final features
    final_features = [feat for feat in hyp_tuning_df.loc[model_order, "final_features"]]

    #Joined training and validation labels.
    y_train_valid = pd.concat((y_train, y_valid))

    #Filtered joined training and validation set based on final features.
    x_train_valid_filtered = pd.concat((x_train, x_valid))[final_features]
    
    #Filtered test set based on final features.
    x_test_filtered = x_test[final_features]

    return (y_train_valid, x_train_valid_filtered, x_test_filtered)


(
  y_train_valid,
  X_train_valid_binned_filtered,
  X_test_binned_filtered
) = data_filter_join(model_selection_df, X_train_binned, X_valid_binned, X_test_binned, y_train, y_valid)

preprocessed = prep_data_export((X_train_valid_binned_filtered, X_test_binned_filtered),
                                    (y_train_valid, y_test),
                                    ("Training_Validation", "Test"),
                                    csv_name = "preprocessed_data")


def final_model_fit_eval(hyp_tuning_df: pd.DataFrame,
                    x_fit: pd.DataFrame, y_fit: pd.Series,
                    model_order: int = 0,
                    save_model: bool = True):
  
    final_model_path = "./models/model_selection/models"
    output_path = "./models/objects_FINAL"

    final_model_name = hyp_tuning_df.loc[model_order, "tuned_model_name"]
    fs_model_name = hyp_tuning_df.loc[model_order, "fs_model_name"]

    #Final model
    with open(os.path.join(final_model_path, f"{final_model_name}_with_{fs_model_name}.h5"), "rb") as mod_load:
        final_model = dill.load(mod_load)

    #Fitting the final model on the joined training and validation set
    final_model.fit(x_fit, y_fit)
    
    if save_model:
        with open(f"{output_path}/final_model_eval.h5", "wb") as mod_save:
            dill.dump(final_model, mod_save)

    return final_model


final_model_eval = final_model_fit_eval(model_selection_df, X_train_valid_binned_filtered, y_train_valid)

def get_final_features(model_selection_df: pd.DataFrame, model_order: int = 0) -> tuple[float, list]:

    final_features = model_selection_df.loc[model_order, "final_features"]
    
    return final_features


final_features = get_final_features(model_selection_df)

opt_threshold = calc_opt_threshold(final_model_eval, X_train_valid_binned_filtered, y_train_valid)
\end{lstlisting}

\subsection{Evaluation}
\begin{lstlisting}[language=Python, basicstyle=\footnotesize\ttfamily]
def conf_mat(model, X: pd.DataFrame, y: pd.Series, threshold: float) -> pd.DataFrame:

    y_scores = pd.Series(model.predict_proba(X)[:, 1])
    y_preds = y_scores.apply(lambda x: 1 if x > threshold else 0)

    confm = pd.DataFrame(
                         confusion_matrix(y, y_preds),
                         columns = ["Predicted - Non-Default", "Predicted - Default"],
                         index = ["Actual - Non-Default", "Actual - Default"]
                        )
    return confm


conf_matrix = conf_mat(final_model_eval, X_test_binned_filtered, y_test, opt_threshold)


def conf_mat_plot(conf_matrix: pd.DataFrame, export:bool = True):

    matplotlib.rcParams["mathtext.fontset"] = "stix"
    matplotlib.rcParams["font.family"] = "STIXGeneral"
    
    plt.figure(figsize = (9, 7))
    #plt.title("Confusion matrix", size = 13, fontweight = "bold")


    ax = sns.heatmap(conf_matrix, annot=True, cmap = "BuPu", fmt = "g", annot_kws = {"size": 18})

    cbar = ax.collections[0].colorbar
    cbar.ax.tick_params(labelsize = 18)  # Set fontsize for color bar

    plt.xticks(size = 18)
    plt.yticks(size = 18)

    plt.tight_layout()


    if export:
        os.makedirs("./plots", exist_ok=True)
        plt.savefig(f"./plots/Confusion_Matrix.jpg", dpi=300)
    plt.show()


conf_mat_plot(conf_matrix)

def evaluation_metrics(model, X: pd.DataFrame, y: pd.Series,
                       threshold: float, target: str = "BAD") -> pd.DataFrame:

    metrics = {
                "F1": f1_score,
                "Precision": precision_score, 
                "Recall": recall_score, 
                "Accuracy": accuracy_score,
                "AUC": roc_auc_score,
                "Somers D": somersd,
                "KS": ks_2samp, 
                "MCC": matthews_corrcoef,
                "Brier Score Loss": brier_score_loss,
                "Log Loss": log_loss
                }


    probs_evs = ["AUC","Brier Score Loss"]
    class_evs = ["Precision", "Recall", "F1", "Accuracy", "MCC"]
    evs_list = []

    y_scores = model.predict_proba(X)[:, 1]
    y_preds = pd.Series(y_scores).apply(lambda x: 1 if x > threshold else 0)
    
    for metric_name, metric in metrics.items():
        if metric_name in probs_evs:
            evs_list.append([metric_name, metric(y, y_scores)])
        elif metric_name in class_evs:
            evs_list.append([metric_name, metric(y, y_preds)])
        elif metric_name == "Log Loss":
            evs_list.append([metric_name, metric(y, model.predict_proba(X))])
        elif metric_name == "Somers D":
            evs_list.append([metric_name, metric(y, y_scores).statistic])
        elif metric_name == "KS":
            X_Y_concat = pd.concat((y, X), axis = 1)
            X_Y_concat["prob"] =  y_scores
            evs_list.append([metric_name, metric(X_Y_concat.loc[X_Y_concat[target] == 1, "prob"],
                                                 X_Y_concat.loc[X_Y_concat[target] == 0, "prob"]).statistic])

    evaluation_df = pd.DataFrame(evs_list, columns = ["Metric", "Score"])

    return evaluation_df


evaluation_metrics(final_model_eval, X_test_binned_filtered, y_test, opt_threshold)


def ROC_curve_plot(model, X: pd.DataFrame, y: pd.Series, export:bool = True):

    matplotlib.rcParams["mathtext.fontset"] = "stix"
    matplotlib.rcParams["font.family"] = "STIXGeneral"

    y_pred = model.predict_proba(X)[:, 1]
    fpr, tpr, _ = roc_curve(y, y_pred)
    auc = roc_auc_score(y, y_pred)
    plt.figure(figsize = (10, 7))
    plt.plot(fpr,tpr,label = f"AUC = {auc*100:.2f}%")
    plt.plot([0, 1], [0, 1], "r--")
    plt.ylabel("True Positive Rate",  size = 18)
    plt.xlabel("False Positive Rate",  size = 18)
    plt.xticks(size = 18)
    plt.yticks(size = 18)
    plt.legend(prop={'size': 16})
    #plt.title("ROC curve", size = 13, fontweight = "bold")
    plt.grid()
    axes = plt.gca()
    axes.spines["top"].set_visible(False)
    axes.spines["right"].set_visible(False)


    plt.tight_layout()

    if export:
        os.makedirs("./plots/", exist_ok = True)
        plt.savefig(f"./plots/ROC_curve_FINAL.jpg", dpi = 300)
    plt.show()


ROC_curve_plot(final_model_eval, X_test_binned_filtered, y_test)


def feature_importance_plot(model, features: list, export:bool = True):

    matplotlib.rcParams["mathtext.fontset"] = "stix"
    matplotlib.rcParams["font.family"] = "STIXGeneral"

    importances = model.feature_importances_

    plot_df = pd.DataFrame({"Feature": features, "Importance": importances})
    plot_df = plot_df.sort_values("Importance", ascending = False)

    plt.figure(figsize = (10, 6))
    sns.barplot(plot_df, x = "Importance", y = "Feature", palette = "BuPu_r")
    #plt.title("Feature Importances", fontsize = 15, fontweight = "bold")

    axes = plt.gca()
    axes.spines["top"].set_visible(False)
    axes.spines["right"].set_visible(False)

    axes.xaxis.label.set_size(18)
    axes.yaxis.label.set_size(18)
    plt.xticks(fontsize = 18)
    plt.yticks(fontsize = 18)

    plt.tight_layout()

    if export:
        os.makedirs("./plots/", exist_ok = True)
        plt.savefig(f"./plots/Feature_Importances.jpg", dpi = 300)
        
    plt.show()


feature_importance_plot(final_model_eval, final_features)


def GBC_SHAP_plot(model, X: pd.DataFrame, export: bool = True):

    matplotlib.rcParams["mathtext.fontset"] = "stix"
    matplotlib.rcParams["font.family"] = "STIXGeneral"


    explainer = shap.TreeExplainer(model, model_output = "probability", data = X)

    shap_values = explainer.shap_values(X)

    #Scaling the SHAP values in order to be between -1 and 1.
    #max_abs_shap = np.max(np.abs(shap_values))
    #scaled_shap_values = shap_values / max_abs_shap

    # Set plot size here
    plot_size = (16, 12)

    # Create a summary plot and specify the plot_size
# Set plot size
    plot_size = (8, 8)

    # Create a summary plot and specify the plot_size
    summary_plot = shap.summary_plot(shap_values,
                                    X.values,
                                    feature_names = X.columns,
                                    show = False,
                                    plot_size = plot_size,
                                    color_bar = True)

    # Increase fontsize of labels and ticks
    plt.gca().xaxis.label.set_size(18)
    plt.gca().yaxis.label.set_size(18)
    plt.xticks(fontsize = 18)
    plt.yticks(fontsize = 18)

    # Find the colorbar object and modify its fontsize
    colorbar = plt.gcf().get_axes()[-1]
    colorbar.tick_params(labelsize = 18)
    colorbar.set_ylabel("Feature Value", fontsize = 18)  # Increase fontsize of the "Feature Value" label

    plt.tight_layout()

    if export:
        os.makedirs("./plots/", exist_ok = True)
        plt.savefig(f"./plots/SHAP_summary_plot.jpg", dpi = 1200)

    plt.show()


GBC_SHAP_plot(final_model_eval, X_test_binned_filtered)


def final_model_fit_deploy(final_model, X: tuple, y: tuple, final_features: list,
                           target = "BAD", export: bool = True):

    df_list = []
    output_path = "./models/objects_FINAL"

    #Joining the features and labels from training, validation and test sets.
    for feat, lab in zip(X, y):
        temp = pd.concat((lab, feat), axis = 1)
        df_list.append(temp)
    dfs = [df for df in df_list]

    #Final model fit.
    final_df = pd.concat(dfs, axis = 0).sort_index()
    X_final = final_df[final_features]
    y_final = final_df[target]

    final_model.fit(X_final, y_final)
    
    #Exporting the final model.
    if export:
        with open(f"{output_path}/final_model_deploy.h5", "wb") as mod_save:
            dill.dump(final_model, mod_save)

    #Final classification threshold
    final_threshold = calc_opt_threshold(final_model, X_final, y_final)

    return (final_model, final_threshold)

(
  final_model_deployment, final_threshold_deployment
) = final_model_fit_deploy(final_model_eval,
                           (X_train_valid_binned_filtered, X_test_binned_filtered),
						   (y_train_valid, y_test),
						   final_features)

def flask_app_input_export(**kwargs):

    with open(r"flask_app\inputs\inputs_flask_app_dict.pkl", "wb") as f:
        dill.dump(kwargs, f)

    with open(r"models\objects_FINAL\inputs_flask_app_dict.pkl", "wb") as f:
        dill.dump(kwargs, f)
		

def lime_explanation(X: pd.DataFrame, export: bool = True):

    explainer = lime.lime_tabular.LimeTabularExplainer(
                                                        training_data = np.array(X),
                                                        feature_names = X.columns,
                                                        class_names = ["non-default", "default"],
                                                        mode = "classification"
                                                        )
    
    return explainer


lime_explainer = lime_explanation(pd.concat((X_train_valid_binned_filtered, X_test_binned_filtered)))


flask_app_input_export(woe_bins = woe_bins,
                       binning_transformator = binning_transformator,
                       final_model = final_model_deployment,
                       threshold = final_threshold_deployment,
                       final_features = final_features,
                       categorical_features = cat_vars,
                       input_df = pd.DataFrame(columns = X_train_binned.columns),
                       lime_explainer = lime_explainer)
\end{lstlisting}

\section{Back--end: Flask Web Application Code (\lstinline{app.py})}
\begin{lstlisting}[language=Python, basicstyle=\footnotesize\ttfamily]
import warnings

warnings.filterwarnings("ignore")
from flask import Flask, render_template, request
import pandas as pd
import numpy as np
import os
import lime
import dill
import matplotlib
import matplotlib.pyplot as plt


app = Flask(__name__)


current_dir = os.path.dirname(os.path.abspath(__file__))
input_file_path = os.path.join(current_dir, "inputs", "inputs_flask_app_dict.pkl")

with open(input_file_path, "rb") as f:
	inputs = dill.load(f)

lime_explainer = inputs["lime_explainer"]    

def lime_plot(input, model, lime_explainer):
	def custom_lime_plot(exp, pos_color: str = "red", neg_color: str = "green"):

		fig, ax = plt.subplots()
		vals = [i[1] for i in exp.as_list()][::-1]
		names = [
			i[0].split(" <=")[0].split("< ")[-1].split(" >")[0] for i in exp.as_list()
		][::-1]
		feat_dict = {
			"JOB": "Job occupancy",
			"REASON": "Reason of loan application",
			"LOAN": "Requested loan amount",
			"MORTDUE": "Amount due on existing mortgage",
			"VALUE": "Current property value",
			"YOJ": "Years at present job",
			"DEROG": "# of major derogatory reports",
			"DELINQ": "# of delinquent credit lines",
			"CLAGE": "Age of the oldest credit line",
			"NINQ": "# of recent credit inquiries",
			"CLNO": "# of credit lines",
			"DEBTINC": "Debt-to-income ratio",
		}
		names = [feat_dict[i] for i in names]
		colors = [pos_color if x > 0 else neg_color for x in vals]
		pos = np.arange(len(vals)) + 0.5

		ax.barh(pos, vals, align="center", color=colors)
		ax.set_yticks(pos)
		ax.set_yticklabels(names)

		return fig

	plt.rcParams["font.family"] = "Segoe UI"

	exp = lime_explainer.explain_instance(
		data_row=input, predict_fn=model.predict_proba
	)

	fig = custom_lime_plot(exp)
	fig.set_size_inches(11, 9)

	ax = plt.gca()
	ax.spines["right"].set_visible(False)
	ax.spines["top"].set_visible(False)
	ax.set_ylabel("Feature", fontsize=21, color="#6c6c6c")
	ax.set_xlabel("Contribution", fontsize=21, color="#6c6c6c")
	ax.tick_params(axis="x", labelsize=20)
	ax.tick_params(axis="y", labelsize=20)

	plt.title("")
	for xticklabel, yticklabel in zip(ax.get_xticklabels(), ax.get_yticklabels()):
		xticklabel.set_color("#6c6c6c")
		yticklabel.set_color("#6c6c6c")

	fig.savefig(
		os.path.join(current_dir, "static", "lime_explanation.png"),
		format="png",
		bbox_inches="tight",
		dpi=300,
		transparent=True,
	)
	plt.close(fig)


@app.route("/")
def home():
	features = inputs["final_features"]
	categorical_features = inputs["categorical_features"]
	return render_template(
		"index.html", variables=features, categorical_features=categorical_features
	)


@app.route("/predict", methods=["POST"])
def predict():
	(
		woe_bins,
		woe_binning,
		model,
		threshold,
		final_features,
		categorical_features,
		input_df,
	) = (input for _, input in inputs.items())

	for feature in input_df.columns:
		if feature in final_features:
			if feature in categorical_features:
				input_df.loc[0, feature] = (
					str(request.form[feature]) if request.form[feature] else np.nan
				)
			elif feature == "LOAN":
				input_df.loc[0, feature] = (
					int(request.form[feature]) if request.form[feature] else np.nan
				)
			else:
				input_df.loc[0, feature] = (
					float(request.form[feature]) if request.form[feature] else np.nan
				)
		else:
			input_df.loc[0, feature] = np.nan

	input_df_woe = woe_binning.transform(input_df, metric="woe")

	for feature in input_df_woe.columns:
		na_woe = woe_bins.query('Variable == @feature and Bin == "Missing"')[
			"WoE"
		].values[0]
		input_df_woe.loc[input_df[feature].isna(), feature] = na_woe

	input_df_woe_FINAL = input_df_woe[final_features]
	pred_score = model.predict_proba(input_df_woe_FINAL)[:, 1]

	result = [
		"Loan application denied" if i > threshold else "Loan application approved"
		for i in pred_score
	]

	lime_plot(input_df_woe_FINAL.iloc[0], model, lime_explainer)















	return render_template(
		"results.html",
		probability_of_default = round(pred_score[0] * 100, 2),
		prediction = result[0],
	)

if __name__ == "__main__":
	app.run(debug = True)	
\end{lstlisting}

\section{Front--end: HTML Code}
\subsection{Application Form (\lstinline{index.html})}
\begin{lstlisting}[language=HTML, basicstyle=\footnotesize\ttfamily]
<!DOCTYPE html>
<html>
	<head>
		<style>
			body {
				font-family: Arial, sans-serif;
				background-color: #f3f3f3;
				color: #444;
			}
			form {
				margin: 20px auto;
				padding: 30px;
				background-color: #ffffff;
				border: 1px solid #e6e6e6;
				box-shadow: 2px 2px 15px rgba(0, 0, 0, 0.1);
				max-width: 700px;
				border-radius: 5px;
			}
			h1 {
				text-align: center;
				color: #333;
				margin-bottom: 20px;
			}

			p {
				text-align: center;
				color: #666;
				margin-bottom: 30px;
			}

			label {
				display: block;
				margin-bottom: 5px;
				color: #555;
			}

			input[type="number"],
			select {
				padding: 10px 20px;
				font-size: 16px;
				border: 1px solid #ccc;
				border-radius: 4px;
				width: 100%;
				box-sizing: border-box;
				margin: 10px 0;
				transition: border 0.3s;
			}

			input[type="submit"] {
				background-color: #4caf50;
				color: white;
				padding: 12px 30px;
				border: none;
				border-radius: 4px;
				cursor: pointer;
				margin-top: 30px;
				width: auto;
				transition: background-color 0.3s;
				display: block;
				margin-left: auto;
				margin-right: auto;
			}

			input[type="submit"]:hover {
				background-color: #45a049;
			}

			select:hover,
			input[type="number"]:hover,
			select:focus,
			input[type="number"]:focus {
				border-color: #4caf50;
			}
			.form-grid {
				display: grid;
				grid-template-columns: repeat(2, 1fr);
				gap: 20px;
			}

			.form-grid > div {
				grid-column: span 1;
			}
		</style>
	</head>
	<body>
		<h1>Default Prediction Application</h1>
		<p><b>Author:</b> Petr Nguyen</p>
		<form method="POST" action="/predict">
			<div class="form-grid">
				
				<div>
					
					<label for="JOB">Job occupancy:</label>
					<select name="JOB">
						<option value=""></option>
						<option value="Mgr">Manager</option>
						<option value="Office">Office worker</option>
						<option value="ProfExe">Professional/Executive</option>
						<option value="Sales">Sales</option>
						<option value="Self">Self-employed</option>
						<option value="Other">Other</option>
					</select>
					<br />
					
					<label for="REASON">Reason of loan application:</label>
					<select name="REASON">
						<option value=""></option>
						<option value="DebtCon">Debt Consolidation</option>
						<option value="HomeImp">Home Improvement</option>
					</select>
					<br />
					
					<label for="LOAN">Requested loan amount:</label>
					<input type="number" name="LOAN" min="0" required /><br />
					
					<label for="MORTDUE">Amount due on existing mortgage:</label>
					<input type="number" name="MORTDUE" min="0" /><br />
					
					<label for="VALUE">Current property value:</label>
					<input type="number" name="VALUE" min="0" /><br />
					
					<label for="YOJ">Number of years at present job:</label>
					<input type="number" name="YOJ" min="0" step="0.1" /><br />
					
					<label for="DEROG">Number of major derogatory reports:</label>
					<input type="number" name="DEROG" min="0" /><br />
					
					<label for="DELINQ">Number of delinquent credit lines:</label>
					<input type="number" name="DELINQ" min="0" /><br />
					
					<label for="CLAGE">Age of the oldest credit line (in months):</label>
					<input type="number" name="CLAGE" min="0" /><br />
					
					<label for="NINQ">Number of recent credit inquiries:</label>
					<input type="number" name="NINQ" min="0" /><br />
					
					<label for="CLNO">Number of credit lines:</label>
					<input type="number" name="CLNO" min="0" /><br />
					
					<label for="DEBTINC">Debt-to-income ratio:</label>
					<input type="number" name="DEBTINC" min="0" step="0.0000001" /><br />
					
				</div>
				
			</div>
			<input type="submit" value="Submit" />
		</form>
	</body>
</html>	
\end{lstlisting}

\subsection{Application Results (\lstinline{results.html})}

\begin{lstlisting}[language=HTML, basicstyle=\footnotesize\ttfamily]
<!DOCTYPE html>
<html>
	<head>
		<title>Default Prediction - Result</title>
		<style>
			body {
				font-family: "Segoe UI", Tahoma, Geneva, Verdana, sans-serif;
				background-color: #f5f5f5;
				color: #333;
			}
			h1 {
				text-align: center;
				color: #4c4c4c;
				font-size: 2.5em;
				margin-top: 50px;
			}
			p {
				text-align: center;
				color: #6c6c6c;
				font-size: 1.5em;
				margin-top: 50px;
			}
			.box {
				background-color: #ccc;
				width: 200px;
				height: 200px;
				display: flex;
				justify-content: center;
				align-items: center;
				font-size: 24px;
				font-weight: bold;
				text-align: center;
				text-decoration: none;
				color: #333;
			}
			img {
				max-width: 100%;
			}
		</style>
	</head>
	<body>
		<h1>Default Prediction Result</h1>
		<p>Result: <b>{{predicted_class}}!</b></p>
		<p>The probability of default is <b>{{ prediction }}%</b></p>

		<img src="{{ url_for('static', filename='lime_explanation.png') }}" 
				alt="LIME Explanation" width="50%" style="display: block; margin: auto;" />
	</body>
</html>	
\end{lstlisting}

\KOMAoptions{paper=portrait,DIV=last}
\restoregeometry
\fancyheadoffset{0pt}