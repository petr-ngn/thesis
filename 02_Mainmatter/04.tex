\chapter{Application of Machine Learning Algorithms}
\label{chap:four}

\section{Repository and Environment Structure}
\section{Data Exploration}
This section is focused on exploration of the analyzed loan dataset, particularly on dataset description, distribution analysis and association analysis, in order to infer potential valuable insights and hypotheses which can be used in the preprocessing or modelling part.
\subsection{Dataset Description}
The analyzed dataset pertains to the HMEQ dataset which contains loan application information and default status of 5,960 US home equity loans. Such dataset was acquired from Credit Risk Analytics.

As can be seen, the dataset contains 12 columns, 11 features and 1 target variable \texttt{BAD} indicating whether the loan was in default (\texttt{1}) or not (\texttt{0}). Amongst the 11 features, there are 9 numeric features and 2 categorical features, namely \texttt{REASON} which contains 2 categories - Debt consolidation (\texttt{DebtCon}) and Home improvement (\texttt{HomeImp}), and \texttt{JOB} which fontains following categories - Administration (\texttt{Offce}), Sales, Manager (\texttt{Mgr}), Professional Executive (\texttt{ProfExe}), Self-employed (\texttt{Self}), and Other.


\begin{table}[!htbp]
	\small
	\setlength{\tabcolsep}{8pt}
	\renewcommand{\arraystretch}{1.3}
	\begin{center}
		\caption[Dataset columns]{Dataset columns}\label{tab:values}
	\begin{tabular}{@{} l p{8cm} l @{}}
	\toprule
	\textbf{Columns} & \textbf{Description} & \textbf{Data type}\\
	\midrule
	BAD & Default status & Boolean \\
	\hline

	LOAN & Requested loan amount & numeric \\
	\hline
	MORTDUE & Loan amount due on existing mortgage & numeric \\
	\hline
	VALUE & Value of current underlying collateral property & numeric \\
	\hline
	REASON & Reason of loan application & categorical \\
	\hline
	JOB & Job occupancy category & categorical \\
	\hline
	YOJ & Years of employment at present job & numeric \\
	\hline
	DEROG & Number of derogatory public reports & numeric \\
	\hline
	DELINQ & Number of delinquent credit lines & numeric \\
	\hline
	CLAGE & Age of the oldest credit line in months & numeric \\
	\hline
	NINQ & Number of recent credit inquiries & numeric \\
	\hline
	CLNO & Number of credit lines & numeric \\
	\hline
	DEBTINC & Debt-to-income ratio & numeric \\
	\bottomrule
	\end{tabular}  
	\end{center}
	\begin{source} \url{http://www.creditriskanalytics.net/datasets-private2.html}\end{source}
\end{table}

\subsection{Distribution Analysis}
\subsection{Association Analysis}

\begin{equation}\label{eq}
	\rho_{spearman} = 1 - \frac{6 \sum_{i=1}^{n} d_{i}^{2}}{n \left(n^{2}-1\right)}
	\end{equation}

\begin{equation}\label{eq}
	r_{X} = \frac{\mu \left( X | Y=1 \right) -\mu_{X}}{\sigma_{X}} \sqrt{\frac{\Pr \left(Y=1\right)}{1-\Pr \left(Y=1\right)}}
	\end{equation}

\begin{equation}\label{eq}
		CV_{X} = \sqrt{\frac{\chi^{2}}{N\left(k-1\right)}}
		\end{equation}

		\begin{equation}\label{eq}
			\phi_{X} = \sqrt{\frac{\chi^{2}}{n}}
			\end{equation}
\section{Data Preprocessing}
\subsection{Data Split and ADASYN Oversampling}
\subsection{Optimal Binning and Weight-of-Evidence Encoding}
\section{Modelling}
\subsection{Feature Selection}
\subsection{Model Selection}
\subsection{Model Building}
\section{Model Evaluation}
\subsection{Confusion Matrix}
\subsection{Metrics Scores}
\subsection{ROC Curve}
\subsection{Learning Curve}
\subsection{SHAP Values}
\section{Machine Learning Deployment}
\subsection{Final Model Building}
\subsection{Web Application}

\section{Itemization and Environments}

Many people use simple n-dash in many occasions -- like this --, where however typographic convention---it looks a bit strange at first sight---requires m-dash. Text text text text text text text text text text text text text text text. Text text text text text text text text text text. Text text text text text text \citet{Haufler2006}. 

Text text text text text text text text text text text text text text text. Text text text text text text text text text text. Text text text text text text. Text text text text text text text text text text text text text text text. Text text text text text text text text text text. Text text text text text text \citet{Wells2001}. Let us describe the following animals:

\begin{description}
\item[Item 1] Text text text text text text text text text text text text text text text. Text text text text text text text text text text. Text text text text text text. Text text text text text text text text text text text text text text text. Text text text text text text text text text text. Text text text text text text.
\item[Item 2] Text text text text text text text text text text text text text text text. Text text text text text text text text text text. Text text text text text text. Text text text text text text text text text text text text text text text. Text text text text text text text text text text. Text text text text text text.
\end{description}

Text text text text text text text text text text text text text text text. Text text text text text text text text text text. Text text text text text text. Text text text text text text text text text text text text text text text. Text text text text text text text text text text. Text text text text text text. See what Edmund Burke said about the duties of a Member of Parliament (Speech To The Electors Of Bristol At The Conclusion Of The Poll, November 3, 1774):

\begin{quotesmall}
It ought to be the happiness and glory of a representative to live in the strictest union, the closest correspondence, and the most unreserved communication with his constituents. Their wishes ought to have great weight with him; their opinion, high respect; their business, unremitted attention. It is his duty to sacrifice his repose, his pleasures, his satisfactions, to theirs; and above all, ever, and in all cases, to prefer their interest to his own. But his unbiased opinion, his mature judgment, his enlightened conscience, he ought not to sacrifice to you, to any man, or to any set of men living. These he does not derive from your pleasure; no, nor from the law and the constitution. They are a trust from Providence, for the abuse of which he is deeply answerable. Your representative owes you, not his industry only, but his judgment; and he betrays, instead of serving you, if he sacrifices it to your opinion.
\end{quotesmall}

Text text text text text text text text text text text text text text text. Text text text text text text text text text text. Text text text text text text.Text text text text text text text text text text text text text text text. Text text text text text text text text text text. Text text text text text text.

\begin{listi}
	\item The first item, the first item, the first item, the first item, the first item, the first item,
	\item and the second item.
\end{listi}

\begin{lista}
	\item The first item, the first item, the first item, the first item, the first item, the first item, 
	\item and the second item.
\end{lista}

Text text text text text text text text text text text text text text text. Text text text text text text text text text text. Text text text text text text. Text text text text text text text text text text text text text text text. Text text text text text text text text text text. Text text text text text text. Text text text text text text text text text text text text text text text. Text text text text text text text text text text. Text text text text text text \citet{Blomstrom2003}. 