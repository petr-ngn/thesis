\chapter{Introduction}
\label{chap:one}


Machine learning is becoming increasingly important in the financial industry as it can bring competitive edge. Many financial institutions have already utilized the machine learning techniques for assessing creditworthiness of the clients and predicting their future behavior \citep{PwC2023}.
As the backbone of most financial institutions, credit risk modelling plays an integral role in ensuring the stability and profitability of these establishments as it primarily revolves around assessing the likelihood that a borrower will default on a debt by failing to make required payments.

Conventional methodologies for credit risk assessment have relied on statistical techniques and rule-based systems, which have been known to deliver sound results.
However, with the exponential growth of data in recent years, financial institutions are confronted with the challenge and opportunity to extract valuable insights from this information deluge.
In an era where data-driven decision making is becoming the norm rather than the exception, the application of machine learning in credit risk modelling is not just an interesting research area, but also a future practical necessity.


\autoref{chap:two} introduces theoretical aspects of both credit risk  and machine learning. Particularly credit risk definition, regulation and credit scoring approaches are described. Moreover, machine learning terminology, machine learning algorithms and evaluation metrics are explained. Furthermore, more advanced machine learning techniques are introduced, such as ADASYN oversampling, Optimal Binning, Bayesian hyperparameter optimization or Forward Sequential Feature Selection.

In \autoref{chap:three} we conduct literature review from both credit risk modelling and machine learning fields. Moreover, we propose five hypotheses either based on HMEQ--based studies \citep{serkan2021bagging, zurada2014classification}, or based on researches related to machine learning application in different sectors \citep{de2023predicting, pintelas2020grey, wu2018accurate}.

Within \autoref{chap:four}, an empirical analysis is performed, particularly, we implement our machine learning framework on the exemplary application scoring data set of US home equity loans (HMEQ).
Such framework applies data exploration techniques and further performs data preprocessing transformations, namely data split, ADASYN oversampling and Optimal Binning.
Moreover, this framework involves hyperparameter tuning, which is performed using Bayesian Optimization, as well as feature selection, which is performed using Forward Sequential Feature Selection.
Subsequently, model selection is utilized based on ranking of the proposed range of evaluation metrics.
afterwards, the final model is recalibrated, evaluated and deployed into a production environment as a web application.

Finally, \autoref{chap:five} concludes the thesis and provides a summary of the findings, as well as discusses the limitations of the study and proposes future research directions.


In this thesis, the main goal is to implement a custom machine learning framework developed in Python, which includes data exploration, data preprocessing, hyperparameter tuning, feature selection, model selection, model recalibration, and model evaluation, and further to develop a custom web application, within which the trained model is deployed, into a production environment, using Flask and HTML with CSS and JavaScript elements.
Such comprehensive machine learning framework is applied on an exemplary, application scoring data set of US home equity loans (HMEQ) acquired from Credit Risk Analytics \citep{baesens2016credit}.
Regardless of the potential discrepancy between such exemplary data set and the current state of the loan market, the objective of this thesis is rather to showcase a versatile machine learning implementation framework that can be effectively employed for diverse data sets, transcending industry boundaries and fulfilling varying analytical requirements.
In this thesis, we employ eight different machine learning algorithms, namely Logistic Regression, Decision Tree, Gaussian Naive Bayes, K-Nearest Neighbors, Random Forest, Support Vector Machine, Gradient Boosting, and Neural Network.
As evaluation metrics in order to assess model's performance, we use F1 score, Precision, Recall, Accuracy, Matthews Correlation Coefficient, Area Under the (ROC) Curve (henceforth AUC), Kolmogorov--Smirnov Distance, Somers' D, Brier Score Loss and Log Loss.



In \autoref{chap:four}, we present very extensive machine learning implementation framework, including a description of repository and environment structure of given implementation. Within the data exploration, we describe the analyzed data, inspect distributions of variables and also perform statistical association analysis in order to infer some relationships between the default and the predictors and between the predictors themselves.
In the data preprocessing, we perform split of data into training set (for model building, hyperparameter tuning and feature selection), validation set (for model selection) and test set (for model evaluation).
Further we perform ADASYN oversampling to balanced skewed default distribution and employ Optimal Binning which bins numerical predictors into interval bins and categorical predictors into categorical group bins, optimally with respect to the default status. Such bins are then transformed using Weight--of--Evidence.


In the final \autoref{chap:five}, we summarize the results of this thesis. In particular, we assess the hypotheses proposed in \autoref{chap:three}, discuss the results of the machine learning implementation in this thesis with the results of Aras \citep{serkan2021bagging} and Zurada \citep{zurada2014classification}, present the key findings of machine learning implementation, outline the author's contribution in this thesis and lastly, propose various future research directions.

