\chapter{Conclusion}
\label{conclusion}

\textbf{TBD}

Recap of theoretical background (credit risk definition, ML terminology, ML algorithms, evaluation and other topics).

Hypotheses and literature review

Recap of the ML framework and repository structure / Dataset description / findings from distribution and association analysis.

Data preprocessing - data split, ADASYN, optbinning.

Bayesian Optimization - approach -> Feature selection algorithm

Model selection algorithm - ranking, weights, youden index, distribution analysis, final model

recalibration for evaluation -> evaluation / model performance + model explainability

model deployment - web application / application form / prediction result / local explainability

Hypotheses results + key findings + future recommendations


\begin{table}[H]
    \small
    \setlength{\tabcolsep}{8pt}
    \centering
    \caption[Hypotheses' Results]{Hypotheses' Results}\label{tab:hypoconclusion}
    \renewcommand{\arraystretch}{1.5}
    \begin{tabular}{c p{10cm} c}
    \toprule
    \textbf{\#} & \textbf{Hypothesis} & \textbf{Rejected} \\
    \midrule
    \hline
    H1 & \textit{The recalibration of the model enhances model performance on HMEQ data set.} & NO \\
    H2 & \textit{Either Neural Network or KNN model outperforms all the models on HMEQ data set.} & YES \\
    H3 & \textit{Black--box models perform better than the white--box models on HMEQ data set.} & NO \\
    H4 & \textit{The longer execution time of a model, the better performance on HMEQ data set.} & YES \\
    H5 & \textit{The main default drivers are the debt and/or delinquency features on HMEQ data set.} & NO \\
    \hline
    \bottomrule
    \end{tabular}
    \vspace{0.35em}
    
    \centering{\begin{source}Author's Results\end{source}}\vspace{-1em}
\end{table}